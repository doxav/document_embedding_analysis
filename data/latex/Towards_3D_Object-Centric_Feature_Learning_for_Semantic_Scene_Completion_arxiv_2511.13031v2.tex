%File: formatting-instructions-latex-2026.tex
%release 2026.0
\documentclass[letterpaper]{article} % DO NOT CHANGE THIS
\usepackage{aaai2026}  % DO NOT CHANGE THIS
\usepackage{times}  % DO NOT CHANGE THIS
\usepackage{helvet}  % DO NOT CHANGE THIS
\usepackage{courier}  % DO NOT CHANGE THIS
\usepackage[hyphens]{url}  % DO NOT CHANGE THIS
\usepackage{graphicx} % DO NOT CHANGE THIS
\urlstyle{rm} % DO NOT CHANGE THIS
\def\UrlFont{\rm}  % DO NOT CHANGE THIS
\usepackage{natbib}  % DO NOT CHANGE THIS AND DO NOT ADD ANY OPTIONS TO IT
\usepackage{caption} % DO NOT CHANGE THIS AND DO NOT ADD ANY OPTIONS TO IT
\frenchspacing  % DO NOT CHANGE THIS
\setlength{\pdfpagewidth}{8.5in}  % DO NOT CHANGE THIS
\setlength{\pdfpageheight}{11in}  % DO NOT CHANGE THIS
%
% These are recommended to typeset algorithms but not required. See the subsubsection on algorithms. Remove them if you don't have algorithms in your paper.
\usepackage{algorithm}
\usepackage{algorithmic}


%%%%%%%%%%%%%%%%%%%%%%%%%%%%%%%%%%%%%%%%%%%%%%%%%%%%%%%%%%%%%%%
\usepackage{multirow}
\usepackage{graphicx}
% \usepackage{wrapfig}
\usepackage[dvipsnames,table]{xcolor}
% \newcommand{\red}[1]{{\color{red}#1}}
% \newcommand{\todo}[1]{{\color{red}#1}}
% \newcommand{\TODO}[1]{\textbf{\color{red}[TODO: #1]}}
\usepackage{float}

\definecolor{White}{rgb}{1.,0.,1.}
\definecolor{first}{rgb}{.8,.0,.0}
\definecolor{second}{rgb}{.0,.6,.0}
\definecolor{third}{rgb}{.0,.0,.8}

\definecolor{ceiling}{RGB}{214,  38, 40}
\definecolor{floor}{RGB}{43, 160, 4}
\definecolor{wall}{RGB}{158, 216, 229}
\definecolor{window}{RGB}{114, 158, 206}
\definecolor{chair}{RGB}{204, 204, 91}
\definecolor{bed}{RGB}{255, 186, 119}
\definecolor{sofa}{RGB}{147, 102, 188}
\definecolor{table}{RGB}{30, 119, 181}
\definecolor{tvs}{RGB}{160, 188, 33}
\definecolor{furniture}{RGB}{255, 127, 12}
\definecolor{objects}{RGB}{196, 175, 214}

\definecolor{car}{rgb}{0.39215686, 0.58823529, 0.96078431}
\definecolor{bicycle}{rgb}{0.39215686, 0.90196078, 0.96078431}
\definecolor{motorcycle}{rgb}{0.11764706, 0.23529412, 0.58823529}
\definecolor{truck}{rgb}{0.31372549, 0.11764706, 0.70588235}
\definecolor{othervehicle}{rgb}{0.39215686, 0.31372549, 0.98039216}
\definecolor{person}{rgb}{1.        , 0.11764706, 0.11764706}
\definecolor{bicyclist}{rgb}{1.        , 0.15686275, 0.78431373}
\definecolor{motorcyclist}{rgb}{0.58823529, 0.11764706, 0.35294118}
\definecolor{road}{rgb}{1.        , 0.        , 1.        }
\definecolor{parking}{rgb}{1.        , 0.58823529, 1.        }
\definecolor{sidewalk}{rgb}{0.29411765, 0.        , 0.29411765}
\definecolor{otherground}{rgb}{0.68627451, 0.        , 0.29411765}
\definecolor{building}{rgb}{1.        , 0.78431373, 0.        }
\definecolor{fence}{rgb}{1.        , 0.47058824, 0.19607843}
\definecolor{vegetation}{rgb}{0.        , 0.68627451, 0.        }
\definecolor{trunk}{rgb}{0.52941176, 0.23529412, 0.        }
\definecolor{terrain}{rgb}{0.58823529, 0.94117647, 0.31372549}
\definecolor{pole}{rgb}{1.        , 0.94117647, 0.58823529}
\definecolor{trafficsign}{rgb}{1.        , 0.        , 0.        }
\definecolor{otherstructure}{rgb}{0.98039215, 0.58823529, 0.}
\definecolor{otherobject}{rgb}{0.19607843, 1.        , 1.        }

% \makeatletter
% \newcommand{\car@semkitfreq}{3.92}
% \newcommand{\bicycle@semkitfreq}{0.03}
% \newcommand{\motorcycle@semkitfreq}{0.03}
% \newcommand{\truck@semkitfreq}{0.16}
% \newcommand{\othervehicle@semkitfreq}{0.20}
% \newcommand{\person@semkitfreq}{0.07}
% \newcommand{\bicyclist@semkitfreq}{0.07}
% \newcommand{\motorcyclist@semkitfreq}{0.05}
% \newcommand{\road@semkitfreq}{15.30}
% \newcommand{\parking@semkitfreq}{1.12}
% \newcommand{\sidewalk@semkitfreq}{11.13}
% \newcommand{\otherground@semkitfreq}{0.56}
% \newcommand{\building@semkitfreq}{14.1}
% \newcommand{\fence@semkitfreq}{3.90}
% \newcommand{\vegetation@semkitfreq}{39.3}
% \newcommand{\trunk@semkitfreq}{0.51}
% \newcommand{\terrain@semkitfreq}{9.17}
% \newcommand{\pole@semkitfreq}{0.29}
% \newcommand{\trafficsign@semkitfreq}{0.08}
% \newcommand{\semkitfreq}[1]{{\csname #1@semkitfreq\endcsname}}

% \newcommand{\car@sscbkitfreq}{2.85}
% \newcommand{\bicycle@sscbkitfreq}{0.01}
% \newcommand{\motorcycle@sscbkitfreq}{0.01}
% \newcommand{\truck@sscbkitfreq}{0.16}
% \newcommand{\othervehicle@sscbkitfreq}{5.75}
% \newcommand{\person@sscbkitfreq}{0.02}
% \newcommand{\road@sscbkitfreq}{14.98}
% \newcommand{\parking@sscbkitfreq}{2.31}
% \newcommand{\sidewalk@sscbkitfreq}{6.43}
% \newcommand{\otherground@sscbkitfreq}{2.05}
% \newcommand{\building@sscbkitfreq}{15.67}
% \newcommand{\fence@sscbkitfreq}{0.96}
% \newcommand{\vegetation@sscbkitfreq}{41.99}
% \newcommand{\terrain@sscbkitfreq}{7.10}
% \newcommand{\pole@sscbkitfreq}{0.22}
% \newcommand{\trafficsign@sscbkitfreq}{0.06}
% \newcommand{\otherstructure@sscbkitfreq}{4.33}
% \newcommand{\otherobject@sscbkitfreq}{0.28}
% \newcommand{\sscbkitfreq}[1]{{\csname #1@sscbkitfreq\endcsname}}
%%%%%%%%%%%%%%%%%%%%%%%%%%%%%%%%%%%%%%%%%%%%%%%%%%%%%%%%%%

%%%%%%%%%%%%%%%%%%%%%%%%%%%%%%%%%%%%%%%%%%%%%%%%%%%%%%%%%%
\usepackage{amsmath}
\usepackage{amssymb}
\usepackage{booktabs}
\usepackage{adjustbox}
\usepackage[dvipsnames]{xcolor}
%%%%%%%%%%%%%%%%%%%%%%%%%%%%%%%%%%%%%%%%%%%%%%%%%%%%%%%%%%
%
% These are are recommended to typeset listings but not required. See the subsubsection on listing. Remove this block if you don't have listings in your paper.
\usepackage{newfloat}
\usepackage{listings}
\DeclareCaptionStyle{ruled}{labelfont=normalfont,labelsep=colon,strut=off} % DO NOT CHANGE THIS
\lstset{%
	basicstyle={\footnotesize\ttfamily},% footnotesize acceptable for monospace
	numbers=left,numberstyle=\footnotesize,xleftmargin=2em,% show line numbers, remove this entire line if you don't want the numbers.
	aboveskip=0pt,belowskip=0pt,%
	showstringspaces=false,tabsize=2,breaklines=true}
\floatstyle{ruled}
\newfloat{listing}{tb}{lst}{}
\floatname{listing}{Listing}
%
% Keep the \pdfinfo as shown here. There's no need
% for you to add the /Title and /Author tags.
\pdfinfo{
/TemplateVersion (2026.1)
}

% DISALLOWED PACKAGES
% \usepackage{authblk} -- This package is specifically forbidden
% \usepackage{balance} -- This package is specifically forbidden
% \usepackage{color (if used in text)
% \usepackage{CJK} -- This package is specifically forbidden
% \usepackage{float} -- This package is specifically forbidden
% \usepackage{flushend} -- This package is specifically forbidden
% \usepackage{fontenc} -- This package is specifically forbidden
% \usepackage{fullpage} -- This package is specifically forbidden
% \usepackage{geometry} -- This package is specifically forbidden
% \usepackage{grffile} -- This package is specifically forbidden
% \usepackage{hyperref} -- This package is specifically forbidden
% \usepackage{navigator} -- This package is specifically forbidden
% (or any other package that embeds links such as navigator or hyperref)
% \indentfirst} -- This package is specifically forbidden
% \layout} -- This package is specifically forbidden
% \multicol} -- This package is specifically forbidden
% \nameref} -- This package is specifically forbidden
% \usepackage{savetrees} -- This package is specifically forbidden
% \usepackage{setspace} -- This package is specifically forbidden
% \usepackage{stfloats} -- This package is specifically forbidden
% \usepackage{tabu} -- This package is specifically forbidden
% \usepackage{titlesec} -- This package is specifically forbidden
% \usepackage{tocbibind} -- This package is specifically forbidden
% \usepackage{ulem} -- This package is specifically forbidden
% \usepackage{wrapfig} -- This package is specifically forbidden
% DISALLOWED COMMANDS
% \nocopyright -- Your paper will not be published if you use this command
% \addtolength -- This command may not be used
% \balance -- This command may not be used
% \baselinestretch -- Your paper will not be published if you use this command
% \clearpage -- No page breaks of any kind may be used for the final version of your paper
% \columnsep -- This command may not be used
% \newpage -- No page breaks of any kind may be used for the final version of your paper
% \pagebreak -- No page breaks of any kind may be used for the final version of your paperr
% \pagestyle -- This command may not be used
% \tiny -- This is not an acceptable font size.
% \vspace{- -- No negative value may be used in proximity of a caption, figure, table, section, subsection, subsubsection, or reference
% \vskip{- -- No negative value may be used to alter spacing above or below a caption, figure, table, section, subsection, subsubsection, or reference

\setcounter{secnumdepth}{0} %May be changed to 1 or 2 if section numbers are desired.

% The file aaai2026.sty is the style file for AAAI Press
% proceedings, working notes, and technical reports.
%

% Title

% Your title must be in mixed case, not sentence case.
% That means all verbs (including short verbs like be, is, using,and go),
% nouns, adverbs, adjectives should be capitalized, including both words in hyphenated terms, while
% articles, conjunctions, and prepositions are lower case unless they
% directly follow a colon or long dash
\title{Towards 3D Object-Centric Feature Learning for Semantic Scene Completion}
\author{
    %Authors
    % All authors must be in the same font size and format.
    Weihua Wang\textsuperscript{\rm 1,2}\equalcontrib, 
    Yubo Cui\textsuperscript{\rm 1}\equalcontrib, 
    Xiangru Lin\textsuperscript{\rm 4}, 
    Zhiheng Li\textsuperscript{\rm 1}, 
    Zheng Fang\textsuperscript{\rm 1,2,3}\thanks{Corresponding author.}
}
\affiliations{
    %Afiliations
    \textsuperscript{\rm 1}Faculty of Robot Science and Engineering, Northeastern University, Shenyang, Liaoning, China\\
    \textsuperscript{\rm 2}National Frontiers Science Center for Industrial Intelligence and Systems Optimization, Shenyang, Liaoning, China \\
    \textsuperscript{\rm 3}The Key Laboratory of Data Analytics and Optimization for Smart Industry(Northeastern University), China \\
    \textsuperscript{\rm 4}The University of Hong Kong \\
    % If you have multiple authors and multiple affiliations
    % use superscripts in text and roman font to identify them.
    % For example,

    % Sunil Issar\textsuperscript{\rm 2}, 
    % J. Scott Penberthy\textsuperscript{\rm 3}, 
    % George Ferguson\textsuperscript{\rm 4},
    % Hans Guesgen\textsuperscript{\rm 5}
    % Note that the comma should be placed after the superscript
    % email address must be in roman text type, not monospace or sans serif
    \{wangweihua,ybcui21, zhli24\}@stumail.neu.edu.cn,
    xiangru.lin@connect.hku.hk,\\
    fangzheng@mail.neu.edu.cn
%
% See more examples next
}

%Example, Single Author, ->> remove \iffalse,\fi and place them surrounding AAAI title to use it
\iffalse
\title{My Publication Title --- Single Author}
\author {
    Author Name
}
\affiliations{
    Affiliation\\
    Affiliation Line 2\\
    name@example.com
}
\fi

\iffalse
%Example, Multiple Authors, ->> remove \iffalse,\fi and place them surrounding AAAI title to use it
\title{My Publication Title --- Multiple Authors}
\author {
    % Authors
    First Author Name\textsuperscript{\rm 1,\rm 2},
    Second Author Name\textsuperscript{\rm 2},
    Third Author Name\textsuperscript{\rm 1}
}
\affiliations {
    % Affiliations
    \textsuperscript{\rm 1}Affiliation 1\\
    \textsuperscript{\rm 2}Affiliation 2\\
    firstAuthor@affiliation1.com, secondAuthor@affilation2.com, thirdAuthor@affiliation1.com
}
\fi


% REMOVE THIS: bibentry
% This is only needed to show inline citations in the guidelines document. You should not need it and can safely delete it.
\usepackage{bibentry}
% END REMOVE bibentry

\begin{document}

\maketitle

\begin{abstract}
% 以视觉为基础的三维语义场景补全近年来受到越来越多的关注,尤其在自动驾驶等场景中具有重要应用价值。当前主流方法大多采用以自车为中心(ego-centric)的建模方式,通过对整个场景的全局特征进行汇聚与扩散来完成预测。然而,这种方法往往忽视了细粒度的物体级信息,尤其在复杂环境下,容易造成语义和几何的模糊。为了解决这一问题,我们提出了 Ocean,一种以物体为中心的预测框架,将场景分解为独立的实例单元,从而提升语义占据预测的准确性。具体地,我们首先采用轻量级的分割模型 MobileSAM 获取场景中的实例 mask;随后,引入 3D Semantic Group Attention 实现在三维空间中以实例为单位的图像特征聚合。此外,为应对分割误差与漏检,我们设计了 Global Similarity-Guided Deformable Attention,通过引导性先验特征实现全局特征交互。最后,我们提出了 Instance-aware Local Diffusion 模块,进一步提升实例特征表达,并动态细化场景表示。
Vision-based 3D Semantic Scene Completion (SSC) has received growing attention due to its potential in autonomous driving. While most existing approaches follow an ego-centric paradigm by aggregating and diffusing features over the entire scene, they often overlook fine-grained object-level details, leading to semantic and geometric ambiguities, especially in complex environments. To address this limitation, we propose Ocean, an object-centric prediction framework that decomposes the scene into individual object instances to enable more accurate semantic occupancy prediction. 
Specifically, we first employ a lightweight segmentation model, MobileSAM, to extract instance masks from the input image. Then, we introduce a 3D Semantic Group Attention module that leverages linear attention to aggregate object-centric features in 3D space. To handle segmentation errors and missing instances, we further design a Global Similarity-Guided Attention module that leverages segmentation features for global interaction. Finally, we propose an Instance-aware Local Diffusion module that improves instance features through a generative process and subsequently refines the scene representation in the BEV space.
Extensive experiments on the SemanticKITTI and SSCBench-KITTI360 benchmarks demonstrate that Ocean achieves state-of-the-art performance, with mIoU scores of 17.40 and 20.28, respectively.
\end{abstract}

% Uncomment the following to link to your code, datasets, an extended version or similar.
% You must keep this block between (not within) the abstract and the main body of the paper.
% \begin{links}
%     \link{Code}{https://aaai.org/example/code}
%     \link{Datasets}{https://aaai.org/example/datasets}
%     \link{Extended version}{https://aaai.org/example/extended-version}
% \end{links}


\section{Introduction}
\label{sec:intro}
% 近年来,3D语义场景补全,或称为语义障碍物预测,作为自动驾驶技术的核心模块取得了巨大的进步。通过输出三维场景的稠密的语义表达,3D SSC任务可以更好的服务于下游的规划和避障。同时,相比于基于点云的方法,基于视觉的算法有更低的成本,引起了研究人员越来越多的关注。通过结合深度预测,基于视觉的方法能够在仅使用单目图像的情况下完成对周围环境的三维感知预测。
Semantic Scene Completion (SSC), also known as semantic occupancy prediction, has made significant progress in recent years. By partitioning the 3D space into voxels and predicting a semantic label for each voxel, SSC generates a dense and structured representation of the 3D environment. This fine-grained and enriched semantic detail can better serve downstream planning in autonomous driving. 
Compared to LiDAR-based approaches, vision-based methods are gaining increasing attention due to their lower cost and accessibility. By incorporating depth estimation, monocular vision-based SSC methods can infer 3D semantic information of the surrounding scene using only RGB images.

% \begin{figure}[t]
%     \centering
%     \includegraphics[width=\linewidth]{imgs/intro.pdf}
%     \caption{(a). Previous ego-centric paradigm. (b) Our proposed 3D object-centric paradigm. Different from the ego-centric paradigm, our approach leverages semantic object priors derived from SAM to guide object-level feature aggregation and diffusion. This strategy mitigates semantic ambiguity, resulting in more distinct and accurate object predictions.}
%     \label{fig:intro}
% \end{figure}

\begin{figure}[t]
    \centering
    \includegraphics[width=\linewidth]{imgs/intro.pdf}
    \caption{Comparison between our object-centric learning guided by MobileSAM and previous scene-level paradigms.
    %(a) Previous ego-centric paradigm. (b) Our proposed 3D object-centric paradigm. In contrast to the ego-centric paradigm, our approach introduces the 2D prior of MobileSAM into 3D feature learning, resulting in more distinct and accurate object-level feature interactions.
    }
    \label{fig:intro}
\end{figure}

To project 2D visual features into 3D space for prediction, most previous methods follow an ego-centric paradigm. This approach uses camera-ego relations to transform 2D features into 3D features and then diffuses these features across the entire scene.
For example, MonoScene~\cite{cao2022monoscene} lifts the multi-scale visual feature to 3D spaces with the camera parameters and utilizes 3D convolution to complete the 3D feature to make predictions. VoxFormer~\cite{li2023voxformer} selects a subset of voxels as \textit{query proposals} based on depth predictions, using cross-attention to aggregate features from the image and then employing self-attention to propagate this aggregated information to the entire 3D scene.
% For example, MonoScene~\cite{cao2022monoscene} first adopts vision input to perform the SSC task. With the camera parameters, they lift the multi-scale visual feature to 3D spaces and utilize 3D convolution to complete the 3D feature to make predictions.
% Subsequently, to achieve global feature fusion, TPVformer~\cite{huang2023tri} conducts feature aggregation through the attention mechanism. 
% Moreover, VoxFormer~\cite{li2023voxformer} employs a two-stage pipeline to perform 3D SSC prediction.
% They first select a subset of voxels as \textit{query proposals} based on depth predictions, using cross-attention to aggregate features from the image, and then employ self-attention to diffuse these aggregated features across the entire 3D scene.
% Most of these previous methods follow an ego-centric learning paradigm that aggregates and diffuses features across the entire scene.
However, despite these achievements, this global paradigm does not adequately distinguish between different voxels, resulting in unintended interactions that usually cause semantic and geometric ambiguity.
Specifically, fusing features from different objects would lead to semantic confusion, while fusing features from empty and occupied areas also leads to geometric confusion.
As illustrated in Figure~\ref{fig:intro}(a), when multiple nearby cars are along the roadside, the empty spaces between the cars are mistakenly assigned features resembling those of the cars due to global fusion.
This makes it difficult to distinguish individual cars and leads to incorrect long-trailing predictions.

% Most previous methods follow an ego-centric learning paradigm that aggregates and diffuses features across the entire scene.
% However, this global approach does not adequately distinguish between voxels, resulting in unintended interactions that cause semantic and geometric ambiguity.
% Specifically, fusing features from different objects would lead to semantic confusion, while fusing features from empty and occupied areas also leads to geometric confusion.
% % weakens the feature of the occupied areas and exaggerates the feature of empty areas.
% As illustrated in Figure.~\ref{fig:intro}(a), when multiple cars are parked closely along the roadside, the regions that should remain empty between the cars acquire similar features to the cars due to the global fusion, making it difficult to distinguish one car from another.
In contrast, object-centric learning paradigm has been explored to emphasize more detailed feature interactions, employing instance queries~\cite{jiang2024symphonize} or Gaussian points~\cite{huang2024gaussianformer} to implicitly represent objects and aggregate features from image features. Compared to the ego-centric learning paradigm, this paradigm demonstrates greater potential in aggregating object details.
% cvpr 审稿意见说这里没有理论支撑, 但是阻碍了高效的特征交互,高效一词加上应该问题不大
However, these methods lack explicit object-level correspondences, hindering effective feature interaction at the object level and resulting in limited performance gains. 
Recently, large visual foundation models such as SAM~\cite{kirillov2023segment} have made significant strides in visual understanding, particularly in capturing fine-grained image details. Building on this, MobileSAM~\cite{zhang2023faster} adopts a lightweight design that enables the practical deployment of SAM in real-world downstream tasks.
This raises the question: \textit{Can these models be leveraged to provide finer details and enhance vision-based SSC predictions?}
%在本文中,我们主要关注实例先验信息对语义场景补全(SSC)任务的影响。直接使用视觉基础模型来提供此类先验信息会面临两个主要问题: 1)掩码先验信息受限于二维图像平面,限制了其进行全面的三维特征交互的能力; 2)二维实例先验信息的引入会制约网络的空间建模能力,导致网络过度依赖二维先验信息,削弱其对场景语义和几何信息的学习能力;不仅如此,先验信息中的误差会严重影响网络的输出结果。
%In this paper, we primarily focus on the impact of instance-level prior information on the Semantic Scene Completion (SSC) task. 
However, leveraging these vision foundation models poses two major challenges: 1) the mask priors are constrained to the 2D image plane, which limits their capacity for comprehensive 3D feature interaction. 2) Mistakes and omissions in the prior masks may result in noticeable performance degradation.
%the introduction of 2D instance priors can hinder the network’s spatial modeling capability, causing it to overly rely on 2D priors and weakening its ability to learn semantic and geometric information of the scene. More critically, errors in the prior information can significantly degrade the network’s output quality.

% 为应对上述挑战,我们提出了一个(Ocean),融合了语义组双注意力模块(SGDAB)和实例感知局部扩散模块(ILD)。SGDAB 结合了 3D 语义组注意力(SGA)与全局相似性引导的可变形注意力(GSGA),其中SGA3D通过引入深度信息,实现三维空间中的物体语义汇聚,而 GSGA 通过在特征层面上利用先验mask,实现了实例引导的信息汇聚,并且可变形机制缓解mask的误差的同时还弥补了3D SGA 依赖前景信息导致的空间信息缺失。ILD 模块利用掩码提取实例级特征,随后动态重建实例级的BEV特征,最后进行局部细化融合生成具有实例感知能力的场景表示。
% 格式 
% 我们提出了xxx, 一种xxx方法
% 它显示地聚合3D场景中的实例信息, 并动态地在空间中扩散这些信息, 为了3D semantic scene completion.
%  ocean包含两个模块, 其中一个模块further incorporatesxxx
%   然后单个模块总结
%   第一个模块介绍+总结
To address these challenges, we introduce ocean, a 3D object-centric feature learning network for semantic scene completion, which integrates a SemGroup Dual Attention (SGDA) block and an Instance-aware Local Diffusion (ILD) module. 
The SGDA block incorporates two key components, 3D Semantic Group Attention (SGA3D) and Global Similarity-Guided Attention (GSGA), to enhance 3D semantic understanding.
% 具体来说, SGA3D模块通过在3D空间中, 对同一实例进行特征交互, 促进了物体级别的语义汇聚
Specifically, the SGA3D module facilitates object-level semantic aggregation by performing feature interactions within individual instances in 3D space, while GSGA leverages prior features to correct mask errors and reduce foreground omissions caused by imperfect masks.
Furthermore, the ILD module leverages instance-level features to reconstruct bird’s-eye view representations, which are then used to refine the entire scene feature for finer-grained semantic representation.

In summary, our contributions are as follows:
\begin{itemize}
    % 为了解决场景的物体级语义和几何模糊, 我们提出了ocean,它显示地汇聚了3D object-level feature, 高效地增强了场景的特性信息
    \item To address object-level semantic and geometric ambiguities in scenes, we propose Ocean, a novel approach that explicitly aggregating 3D object-level features and effectively enhancing scene feature representations.
    \item To comprehensively utilize MobileSAM priors, we propose SGDA and ILD module, which are designed to aggregate object-centric features and diffuse instance-aware information throughout the scene, respectively.
    \item Experiments on the SemanticKITTI and SSC-Bench-KITTI360 show that Ocean achieves state-of-the-art performance with mIoU of 17.40 and 20.28, respectively.
\end{itemize}
%-------------------------------------------------------------------------


\section{Related Works}
\label{sec:related_works}
\subsection{Vision-based 3D Perception}
The perception system plays a crucial role in autonomous driving. Compared to LiDAR-based methods, vision-based methods~\cite{philion2020lift, li2022bevformer} offer advantages in terms of lower cost and easier deployment, thus have garnered increasing attention.
% By lifting the vision feature to the 3D space, vision-based methods could also achieve good perception performance. 
LSS~\cite{philion2020lift} lifts 2D image features into 3D by predicting a per-pixel depth distribution and computing its outer product with the features.
However, LSS is sensitive to depth. In contrast, 
BEVformer~\cite{li2022bevformer} directly defines grid-shaped BEV queries to aggregate the BEV features. 
FB-BEV~\cite{li2023fb} analyzes the limitations of forward and backward projection methods and proposes a unified forward-backward paradigm to address them.
Nowadays, with the development of visual research, visual perception has made tremendous progress in detection~\cite{qi2024ocbev, zhang2025geobev}, segmentation~\cite{lu2025toward}.
%, and mapping~\cite{zhang2024enhancing, chang2024bevmap}.

\subsection{Semantic Scene Completion}
MonoScene~\cite{cao2022monoscene} is the first to leverage monocular images for the 3D SSC task. It lifts 2D visual features to 3D space and employs a 3D U-Net to extract voxel features for the final prediction. 
Moreover, TPVFormer~\cite{huang2023tri} introduces a tri-perspective view (TPV) and utilizes cross-view hybrid attention to enable interaction of features across the TPV planes. 
OccFormer~\cite{zhang2023occformer} decomposes 3D feature processing into local and global scales to enable long-range and dynamic interactions.
Voxformer~\cite{li2023voxformer} incorporates depth priors and adopts a sparse-to-dense approach for 3D feature interactions. 
Building upon these works,
MonoOcc~\cite{zheng2024monoocc} improves 3D occupancy via cross-attention and temporal distillation.
CGFormer~\cite{yu2024context} addresses query ambiguity with a context and geometry aware design.
LOMA~\cite{cui2025loma} uses a VL-aware generator and Tri-plane Fusion to boost language integration and global 3D modeling.
%To alleviate the heavy computation introduced by dense voxel representations,
To reduce the heavy computation burden,
SparseOcc~\cite{liu2024fully} decouples semantics and geometry, introducing a sparse and efficient framework.
OctreeOcc~\cite{lu2024octreeocc} addresses dense-grid limitations by adaptively utilizing octree representations.
GaussianFormer~\cite{huang2024gaussianformer} first employs sparse gaussian representations for feature aggregation.
Different from the above methods, Symphonies~\cite{jiang2024symphonize} uses learnable instance queries to iteratively aggregate 3D features in an object-centric pipeline.
This paper adopts object-centric feature learning but uniquely leverages object priors for explicit object-level interactions.




\subsection{Object-Centric Learning}
%Object-centric learning representation, recognized as a promising approach to system generalization, is gaining increasing attention.%
%一个视觉场景可以由多个不同的物体进行表征%
% Object-centric learning aims to represent the entire scene by utilizing multiple distinct objects within it.
% AIR~\cite{eslami2016attend} first represents each object by appearance, position, and presence. The appearance is modeled using a standard VAE~\cite{kingma2013auto}, while the position is refined through a Spatial Transformer~\cite{jaderberg2015spatial}. Finally, these components predicted to be present are combined to generate the final image.
% Afterward, SQAIR~\cite{kosiorek2018sequential} extends AIR into a sequential model to detect and track objects over time. 
% SPAIR~\cite{crawford2019spatially} introduces a grid-based spatial attention mechanism designed to handle scenes with densely arranged objects.
% Object-centric learning seeks to represent an entire scene by leveraging the features of each unique object within it. AIR~\cite{eslami2016attend} models each object based on appearance, position, and presence: appearance is captured through VAE~\cite{kingma2013auto}, position refined with Spatial Transformer~\cite{jaderberg2015spatial}, and predicted components integrated for the final scene representation. SQAIR~\cite{kosiorek2018sequential} extends AIR by introducing a sequential model, enabling object detection and tracking over time. SPAIR~\cite{crawford2019spatially} further incorporates a grid-based spatial attention mechanism, tailored for densely populated scenes.
% Different from AIR, MONET~\cite{burgess2019monet}, IODINE~\cite{greff2019multi}, and GENESIS~\cite{engelcke2019genesis} use spatial mixture models to represent visual scenes, focusing on object structure on a global scale. However, these methods are usually inefficient in terms of computation. 
% To address this issue, Slot Attention~\cite{locatello2020object} develops a grouping strategy that decomposes input features into distinct slot representations through an iterative attention mechanism.
% Building on this, SVAi~\cite{kipf2021conditional} integrates optical flow to capture temporal dynamics in videos, while SLATE~\cite{singh2021illiterate} facilitates zero-shot imagination through learned slot representations.
% Nevertheless, models that predefine the number of slots often fail to account for object count variability. AdaSlot~\cite{fan2024adaptive} addresses this limitation by dynamically adjusting the number of slots based on instance complexity. In this paper, inspired by slot-based methods~\cite{locatello2020object, fan2024adaptive}, we propose an instance-aware feature diffusion mechanism for 3D object-centric learning.

Object-centric learning aims to represent the entire scene by utilizing multiple distinct objects within it. AIR~\cite{eslami2016attend} models each object using appearance, position, and presence, and these components are combined to form the final scene. 
In contrast, MONET~\cite{burgess2019monet} use spatial mixture models for scene representation, but are computationally inefficient. To solve this, Slot Attention~\cite{locatello2020object} develops a grouping strategy to create distinct slot representations.
However, slot counts can’t handle varying object numbers. AdaSlot~\cite{fan2024adaptive} dynamically adjusts slots based on instance complexity. Inspired by slot-based methods, we propose an instance-aware feature diffusion for 3D object-centric learning.
\begin{figure*}[t]
    \centering
    \includegraphics[width=\linewidth]{imgs/framework.pdf}
    \caption{Overview of the proposed Ocean architecture. Given the monocular image as input, we first extract visual features using an image encoder and lift them into 3D space following LSS. To enable object-centric feature learning, we segment the scene using MobileSAM and design the SGDA block to aggregate features through both local and global attention. Furthermore, we propose the ILD module to refine the overall scene representation by incorporating instance-level features.}
    %Overview of the proposed Ocean architecture. Given the monocular image as input, we first extract visual features using an image encoder and lift them into 3D space following LSS. To enable object-centric feature learning, we first segment the scene using MobileSAM and then design the SGDA block to aggregate features based on both local and global attention mechanisms. Furthermore, we introduce the ILD module, which refines the overall scene representation by incorporating instance-level features. Finally, a prediction head takes the refined features to produce the final output.}
    \label{fig:framework}
\end{figure*}
% v1
\section{Methods}
\subsection{Overview}
% For the SSC task, method needs to predict the category of each 3D voxel in the 3D space based on the input monocular image.
The overall architecture of the proposed method is shown in Figure~\ref{fig:framework}. 
Given a monocular image $\mathcal{I} \in \mathbb{R}^{3 \times H \times W}$, the goal of camera-based 3D occupancy prediction is to reconstruct a 3D occupancy scene $\mathcal{O} \in \mathbb{R}^{X \times Y \times Z \times (M + 1)}$, where $M$ denotes the number of semantic classes.
% Given a monocular image $\mathcal{I} \in \mathbb{R}^{H \times W \times 3}$,
First of all, we employ an image encoder to extract multi-scale visual features $\{\mathcal{F}^{(s)} \in \mathbb{R}^{\frac{H}{s} \times \frac{W}{s} \times C_1^{(s)}}\}$ at three different downsampling ratios $s \in \{4, 8, 16\}$.  
Then, each scale-specific feature $\mathcal{F}^{(s)}$ is further processed by two separate modules to predict the corresponding depth distribution $\mathcal{D}^{(s)} \in \mathbb{R}^{\frac{H}{s} \times \frac{W}{s} \times D}$ and context feature $\mathcal{X}^{(s)} \in \mathbb{R}^{\frac{H}{s} \times \frac{W}{s} \times C_2}$, where $D$ is the number of discretized depth bins. 
We compute the outer product between the predicted depth distribution $\mathcal{D}$ and the context feature $\mathcal{X}$, and then apply voxel pooling~\cite{philion2020lift} to lift the 2D image features into a 3D voxel representation $\mathcal{V} \in \mathbb{R}^{x \times y \times z \times C_2}$, where $x, y, z$ denote the spatial dimensions of the 3D scene.  
This lifting operation is performed using the features at scale $s = 8$, which provides a good trade-off between spatial resolution and computational efficiency.
Following previous works~\cite{li2023voxformer, yu2024context}, we also use a pre-trained network~\cite{shamsafar2022mobilestereonet} to predict the depth map, and then select the 3D voxel queries $\mathcal{Q}\in \mathbb{R}^{N\times C_2} (0 \leq N \leq h \times w \times z)$, called query proposals, based on the depth prediction as follows:
\begin{equation}
    \mathcal{Q}=\mathcal{V}[M]
\end{equation}
where $M$ is a binary mask indicating whether the voxel is occupied based on the depth prediction.
Subsequently, to incorporate the object-level mask guidance from MobileSAM into the SSC task, we propose the SemGroup Dual Attention (SGDA) block, which aggregates query proposals with visual features at the object level. 
Furthermore, we introduce the Instance-aware Local Diffusion (ILD) module, which leverages instance priors to refine scene features through localized feature fusion.
% Furthermore, we extend the proposed SGA to 3D space, performing SGA3D and leading to 3D-aware feature aggregation. 
% Subsequently, based on the utilization of SAM~\cite{kirillov2023segment}, we propose the \textbf{3D Semantic Group Attention} (SGA3D) to effectively aggregate the 2D visual feature to the 3D query proposals in an object-centric manner.
% Additionally, we introduce the \textbf{Instance-aware Local Diffusion} (ILD) module to first convert the 2D prior to the 3D feature and then locally refine the aggregated scene feature with instance information.
% disperse the aggregated features to surrounding voxels with instance features. 
% This approach enables a complete reconstruction of the 3D scene representation based on instance features, thereby enhancing overall scene understanding.
% Finally, through a 3D prediction head, we achieve the semantic scene completion task.
Finally, the Semantic Scene Completion prediction is generated through a 3D prediction head.
% achieve the semantic scene completion task.

\begin{figure}[t]
    \centering
    \includegraphics[width=0.8\linewidth]{imgs/SGDAB.pdf}
    \caption{The details of SemGroup Dual Attention Block.
    }
    \label{fig:encoder}
\end{figure}


\subsection{SemGroup Dual Attention Block} 
% 为了解决先前全局的特征交互导致的物体级信息模糊, 本文尝试以物体为中心的语义场景补全, 为此我们设计了
To mitigate the ambiguity of object-level representations introduced by the ego-centric learning paradigm, we propose a novel object-centric framework. 
As shown in Figure~\ref{fig:encoder}, our approach introduces the SGDA block, which includes a 3D Semantic Group Attention module for object-centric local aggregation and a Global Similarity-Guided Attention module for context global interaction. 
This dual attention mechanism effectively captures both fine-grained local details and global contextual cues, significantly enhancing the quality and completeness of scene representations.

\subsubsection{Semantic Group Attention.}
% \label{sec:Semantic Group Attention}
%Recently, SAM has garnered significant attention for its robust zero-shot capabilities and fine-grained segmentation results. In this module, to perform an object-centric learning paradigm, we utilize SAM to provide the object prior, guiding the object-level feature aggregation. 
% To enable object-centric feature aggregation, it is essential to group the projection points and image pixels at the object level. However, grouping at the feature level is challenging due to the limited information provided by the query proposal. Recently, SAM has garnered significant attention for its robust zero-shot capabilities and fine-grained segmentation results. Therefore, we use SAM to facilitate our object-level feature aggregation. 
% The \textit{everything} segmentation of SAM assigns each pixel in the image to its respective instance mask.
\begin{figure}[t]
    \centering
    \includegraphics[width=0.8\linewidth]{imgs/semanticG.pdf}
    \caption{The Semantic Grouping. 3D query proposals are projected onto the image plane, assigned instance IDs via nearest-neighbor sampling, and clustered with image pixels of the same instance for aggregation using linear attention.
    }
    \label{fig:SG}
\end{figure}
% 如图3所示,我们首先使用SAM中的"everything"分割对输入图像中的每个像素进行分割,得到mask S.每一个像素的类别用不同的instance id表示,如果像素没有被分配实例,则id值为0。之后,我们根据visual encoder的降采样率同样将mask进行降采样。具体来讲,假设降采样率为k,则我们将S分为多个k乘k的grid,并统计每个grid中所有instance id出现的次数,将频率最高的instance id作为整个grid的instance id,以此完成Mask的降采样。

% 这里加上一句给定一张图像, 我们需要将图像中的场景分解为一个一个的实例,然后再实例内部进行特征交互, 避免了全局带来空间模糊.
Given a set of query proposals $\mathcal{Q} \in \mathbb{R}^{N \times C_2}$, our goal is to leverage visual features to enhance their representation for perception improvement. To this end, we first project the 3D query proposals onto the image plane using the camera's intrinsic and extrinsic matrices.
For a query located at $(x, y, z)$ in the ego-centric coordinate system, its corresponding 3D pixel coordinate $(u, v, d)$ in the image space is obtained as follows:
\begin{equation}~\label{equ:depth_compute}
d\left[\begin{array}{c}
u \\
v \\
1
\end{array}\right]=K\cdot \left[R|t\right]\cdot\left[\begin{array}{l}
x \\
y \\
z \\
1
\end{array}\right]
\end{equation}
where $K$ is the intrinsic matrix and $[R|t]$ are the camera extrinsic matrices, $d$ is the depth of the projected pixels.

% 经过投影后,得到了与图像同2D平面的3D proposals。 随后,我们用这些proposals在图像平面中汇聚与之对应的实例图像特征, 实现以物体中心的学习范式。 然而, 获得图像的实例信息不是一件容易的事, 并且在同一实例的图像特征和proposal他们的进行特征学习也是一大难题。
After projection, we obtain 3D proposals that are aligned with the 2D image plane. These proposals are then used to aggregate corresponding instance-level image features on the image plane, facilitating an object-centric learning paradigm. However, acquiring accurate instance information from the image is inherently challenging, and learning features between image features of the same instance and their corresponding proposals remains a non-trivial task.
%To enable object-centric feature aggregation, it is essential to group the projection points and image pixels at the object level.
% 随着视觉基础模型的加入, 场景分割任务得到了蓬勃的发展, 我们选择了性能和计算量比较平衡的MobileSAM来获取场景中的实例信息, 并设计了巧妙的对齐方式实现了同一实例信息的汇聚。

% With the rapid development of vision foundation models, scene segmentation has witnessed significant advancements. In this work, we leverage MobileSAM, which achieves a favorable balance between segmentation performance and computational efficiency, to extract instance-level information from the scene.
% Specifically, as shown in Figure. ~\ref{fig:SG}, we first apply the \textit{Automatic Mask} segmentation of MobileSAM to label each pixel with an instance ID. 
% This process generates a mask $\mathcal{S} \in \mathbb{R}^{H \times W \times 1}$ that indicates the instance ID associated with each pixel.
% For pixels not assigned to any instance, the instance mask value is set to $0$. We then downsample the mask $\mathcal{S}$ according to the downsampling rate of the visual encoder to align with the multi-scale feature resolutions, resulting in the scaled masks $\{\mathcal{S}_i\}^s_{i=0}$.
% Furthermore, we use the nearest-neighbor approach to assign an instance ID to each projection point based on its projected coordinates $(u, v)$ to group the projection points with pixels. Pixels and projection points sharing the same instance are grouped into the same clusters. 
% %投影点如果是背景的一部分,那么它将会被标记并且不会执行特征传播%
% In addition, if the nearest pixel assigned to a projection point is part of the background, that projection point will be excluded from the feature aggregating process.
To this end, we adopt MobileSAM to extract per-pixel instance masks with high efficiency. As illustrated in Figure~\ref{fig:SG}, MobileSAM labels each pixel with an instance ID, generating a mask $\mathcal{S} \in \mathbb{R}^{H \times W \times 1}$ where pixels labeled as 0 denote background. Each 3D projection point is mapped to the image plane and assigned the instance ID of its nearest pixel.  
We use the nearest-neighbor approach to assign an instance ID to each projection point based on its projected coordinates $(u, v)$ to group the projection points with pixels. Pixels and projection points sharing the same instance are grouped into the same clusters. Projection points associated with background pixels are excluded from this process.

% 在目标检测和语义分割任务中, 多尺度信息一直是一个关键的要素。因此, 我们首先对mask进行下采样, 并利用下采样的mask对相同的instance id的多尺度图像特征进行堆叠, 避免了低特征分辨率下物体丢失和高特征分辨率下感受野低的问题.
Meanwhile, higher-level features contain rich semantic information, while lower-level features preserve finer-grained texture details. 
For this purpose, we first downsample the instance masks to match the resolutions of the multi-scale features. 
We then group features across different scales based on the downsampled masks $\{\mathcal{S}_i\}^s_{i=0}$, and apply an attention mechanism within each group,  where the query proposals serve as the query $Q$ and the grouped pixel-level features serve as the key $K$ and value $V$, thereby aggregating multi-scale image features corresponding to the same instance ID. 
%This strategy effectively alleviates the loss of object-level information due to low-resolution features, while addressing the limited receptive field inherent in high-resolution features.
To further improve speed and memory efficiency, we employ scattered linear attention~\cite{he2024scatterformer} 
for feature aggregation, formulated as follows:
% To further improve efficiency in terms of speed and memory consumption, we adopt linear attention~\cite{he2024scatterformer} within each cluster, formulated as follows:
% \begin{equation}
%     \tilde{Q}=\frac{\varphi(Q) \sum_{i=1}^{m} \varphi(K_i)^{T} V_i}{\varphi(Q) \sum_{i=1}^{m} \varphi(K_i)^{T}}
% \end{equation}
\begin{equation}
\tilde{Q}=\text{Concat}\left[
\frac{\varphi(Q^j) \sum_{i=1}^{m^j} \varphi(K_i^j)^{T} V_i^j}
     {\varphi(Q^j) \sum_{i=1}^{m^j} \varphi(K_i^j)^{T}}
\right]^M_{j=1}
\end{equation}
where $M$ denotes the number of instances, $m^j$ is the number of pixels in the cluster, and $\varphi(\cdot)$ denotes the kernel function.



\subsubsection{3D Extension with Depth Similarity.}
With the proposed SGA, we integrate the MobileSAM into SSC and perform object-centric learning.
However, this integration can only fuse features in 2D image space, limiting the model's ability to perceive features in 3D space, especially geometric perception. 
Meanwhile, due to the 2D limitation, the prior could not guide the 3D feature diffusion to capture detailed object information.
Thus, lifting the 2D prior to 3D space is essential for enhancing both semantic and geometric perception.
% Since the query proposals $\mathcal{Q}$ originate from the 3D space and the final predictions are also made in the 3D domain, lifting the 2D prior to 3D spaces is necessary for both semantic and geometric perception. 

Inspired by DFA3D~\cite{DFA3D}, we extend SGA to 3D space by integrating depth information to enhance feature aggregation, particularly to capture richer geometric details. Specifically, within each cluster, the depth of grouped pixels is obtained from the predicted depth map $\mathcal{D}^{(s)}$, while the depth of grouped proposals is computed according to Equation~\ref{equ:depth_compute}.
However, pixel depth is represented as an uncertain probability distribution, whereas proposal depth is a deterministic computed value. This discrepancy hinders direct similarity computation between them.

Luckily, we observe that for each pixel, its predicted depth is represented by a softmax probability over multiple depth bins within a given depth range, which can also be interpreted as the pixel’s similarity to each depth bin. Meanwhile, we can also determine the corresponding depth bin for each projected depth of the proposal. 
% Therefore, we can directly retrieve the predicted depth probability of the pixel within the depth bin associated with the projected depth as the depth similarity $A \in \mathbb{R}^{m\times n}$. 
Therefore, we can use the depth bin as a bridge to generate the depth similarity between the proposals and the pixels.
Specifically, for a cluster of $n$ pixels and $m$ proposals, we first gather the depth distribution of the pixels from the depth prediction. Next, for each proposal, we compute its projected depth value and gather the corresponding depth bin. We then extract the corresponding probability values from the pixel depth distribution, resulting in an $m \times n$ depth similarity matrix. 
This approach not only effectively represents the depth similarity between proposals and pixels but also fully leverages the predicted $\mathcal{D}$. Finally, we achieve the SGA3D as follows:
% \begin{equation}
%     \tilde{Q}=\frac{\varphi(Q) \sum_{i=1}^{m}A \sum_{i=1}^{m} \varphi(K_i)^{T} V_i}{\varphi(Q) \sum_{i=1}^{m} \varphi(K_i)^{T}}
% \end{equation}
\begin{equation}
\tilde{Q}=\text{Concat}\left[
\frac{\varphi(Q^j) A^j \sum_{i=1}^{m^j} \varphi(K_i^j)^{T} V_i^j}
     {\varphi(Q^j) \sum_{i=1}^{m^j} \varphi(K_i^j)^{T}}
\right]^M_{j=1}
\end{equation}
where $A^j$ represents the depth similarity in the $j$-th cluster.

\subsubsection{Global Similarity-Guided Attention.}
Benefiting from the SGA3D module, our method enables instance-level feature aggregation in 3D space. However, SGA3D focuses on local feature interactions within individual instances, and the masks generated by MobileSAM may suffer from inaccuracies and missing instances. These limitations can significantly impact the overall prediction performance.
%具体来说,为了保持object-centric paradim,并且缓解pior mask不准确带来的影响,我们引入mobilesam的特征作为引导,用可变形机制汇聚图像的实例特征。 具体来说,
% 我们计算proposals与经过采样后的mobilesam的特征相似度,再通过这个相似度去筛选出同一实例的特征,增强实例信息的同时,可变形机制还突破了分割模型的空间限制。

To address these limitations while maintaining the object-centric paradigm, we introduce the Global Similarity-Guided Attention (GSGA) module. It dynamically aggregates global image features to query proposals via a deformable attention mechanism, guided by global features from MobileSAM to reduce the impact of inaccurate masks.
Specifically, we compute the similarity between each query proposal and the MobileSAM intermediate features sampled at deformable offsets, and use these scores to filter and emphasize features belonging to the same object instance. This instance-aware guidance improves the relevance of feature aggregation, while the deformable mechanism enables the model to flexibly attend to informative regions beyond rigid segmentation boundaries. The final aggregated representation for each proposal is computed as:
% 通过 3D SGA 模块,体素特征已能够有效聚合来自图像的实例级语义信息。然而,鉴于 GroundSAM 分割结果可能存在误差,且3D SGA 模块移除了 ID 为 0 的实例特征,模型在一定程度上仍对前景掩码存在依赖。为减轻这一依赖并进一步挖掘图像的多尺度语义信息,我们提出了一种全局相似度引导的可变形注意力模块(Global Similarity-Guided Deformable Attention, GSGA)。
% 与仅在同一实例掩码内执行局部特征交互的 3D SGA 方法不同,GSGA 模块不仅具备通过可变形注意力机制从全局图像特征中动态聚合信息至查询点(query)的能力,还引入了体素特征与图像特征之间的全局语义相似度引导机制,以增强注意力聚合过程中的语义一致性与判别性。前者使网络突破了 GroundSAM 掩码的空间限制,能够在一定程度上纠正由分割误差引起的语义偏差,并弥补 3D SGA 在仅依赖前景区域时可能导致的信息缺失;后者通过构建全局语义相似度分布,对可变形采样获得的图像特征进行加权重构,引导模型在特征聚合过程中强化语义一致的区域,抑制无关噪声。
% 数学表达式如下:
\begin{equation}
\hat{Q} = \sum_k\left(G_k  W \mathcal{F}(p_q + \Delta p_k)\right) \odot A_k 
\end{equation}
Here, $\odot$ denotes element-wise multiplication, $k$ is the number of sampling points, $G_k$ denotes the similarity matrix between the query and the MobileSAM feature, $A_{k}$ represents the learnable attention weight at the $k$-th sampling location for a given query, and $\Delta p_{k}$ is the offset applied to the query position $p_q$. $W$ denotes the projection weight matrix.
% 这里, $\odot$表示逐元素相乘, $K$表示采样的点数, $A_{qk} $代表基于query的在采样点$k$的可学习的权重, $\Delta p_{qk}$是应用在$p_q$的偏移, $W$表示投影权重. $o_{gs}$表示GroundSAM特征经过采样和线性投影得到的特征.
%This design enables the model to leverage the semantic supervision provided by segmentation masks without directly relying on the foreground masks themselves. It also introduces a degree of fault tolerance, mitigating the impact of incomplete or inaccurate GroundSAM segmentation in certain regions. Overall, the GSGA module achieves efficient cross-modal and cross-scale fusion, providing richer and more robust image semantics for subsequent 3D representation learning.
% 这种设计使得模型在不直接依赖分割前景掩码的前提下,仍能利用其提供的语义监督,同时具备一定的容错能力,能够缓解 GroundSAM 分割结果在部分区域缺失或不准确带来的影响。整体上,GSGA 模块实现了跨模态、跨尺度的高效融合,并为后续的 3D 表征学习提供了更为丰富和鲁棒的图像语义信息。
\begin{figure}[t]
    \centering
    \includegraphics[width=0.85 \linewidth]{imgs/instance_decoder.pdf}
    \caption{The details of the Dynamic Instance Decoder. Given the instance features, we reconstruct them into the scene-level BEV representation using a transposed convolutional decoder. Furthermore, we employ Gumbel Softmax to enable dynamic instance selection.}
    \label{fig:instance}
\end{figure}

\subsection{Instance-aware Local Diffusion Module}
Although the proposed SGDA block allows 3D query proposals to capture rich image features, many voxels still lack sufficient semantic information due to projection limitations. To address this, we introduce an Instance-aware Local Diffusion module that enhances spatial consistency by propagating features across the scene.

\subsubsection{Dynamic Instance Decoder.} 
To effectively leverage prior masks while preserving the instance-centric design, we sum grouped features across multiple scales based on instance masks to obtain aggregated instance-level representations. This simple summation incurs minimal computational cost and, due to its sensitivity to instance size, effectively preserves spatial characteristics at the instance level.
% 然而,在动态且复杂的场景中,从独立的实例特征直接生成精细的BEV特征较为困难。因此,我们采用了生成式方法,通过SGDA的体素特征作为引导,动态地生成富含实例信息的BEV特征,如图2所示。
% 具体来说, 我们使用轻量级的反卷积块,从实例特征中生成生成维度为xy(c+1)的特征。

Given the extracted instance features, we further reconstruct corresponding BEV representations to enhance spatial and semantic understanding.
However, deriving fine-grained BEV features from independent instance representations is still a challenging task in complex and dynamic scenarios. To address this, we employ a generative strategy, using the voxel features from SGDA as guidance to dynamically generate BEV features enriched with instance information, as illustrated in Figure~\ref{fig:framework}. Specifically, we employ a lightweight deconvolution block to generate features with a dimension of $x\times y \times (C_2+1)$ from the instance features. As shown in Figure~\ref{fig:instance}, we also use a two-layer MLP along with Gumbel-Softmax to predict the one-hot decision $\mathcal{Z}$ for each instance feature.
Finally, we apply Softmax to normalize $\alpha$ based on $\mathcal{Z}$, then combine the weighted BEV features to reconstruct the scene BEV representation. The process could be represented as follows:
\begin{align}
    w_l=\frac{\exp(\alpha_l)}{\sum_{l=1}^L \exp(\alpha_l)}&, \hat{w}_l=\frac{\mathcal{Z}_l w_l}{\sum_{l=1}^L \mathcal{Z}_l w_l+\epsilon}\\ 
    \hat{\mathcal{P}} = \sum_{l=1}^{L}& \mathcal{P}_l\odot \hat{w}_l
\end{align}
where $L$ is the number of instances, $\epsilon$ is a small positive value for stable computation, $\mathcal{P}_i$ is $i$-th BEV feature.
% 为了更好的生成bev特征, 我们采用了基于Slot方法的重建损失, 损失函数为:
Furthermore, inspired by the Slot-based method \cite{fan2024adaptive}, we introduce a reconstruction loss to constrain the generated BEV features, enhancing their quality and accelerating convergence. The loss function is defined as follows:
\begin{equation}
\mathcal{L}_{\text{recon}} = \sum_{i=1}^{N} \left\| \hat{\mathcal{P}}^{(i)} - \mathcal{F}_{bev}^{(i)} \right\|_2^2
\end{equation}
where $N$ represents the total number of elements , and $\mathcal{F}_{bev}^{(i)}$ denotes the feature corresponding to the Bird's Eye View (BEV) representation of the entire scene.

\subsubsection{Local Attention Refinement.}
Given the instance-aware BEV features, we use them to refine the aggregated BEV features.
% Given the aggregated BEV features and the instance-aware BEV features, we further fuse them to obtain the final scene feature.
Typically, neighboring voxels contain similar features. Therefore, we use window attention~\cite{liu2021swin} to refine the BEV features in a local manner. Specifically, we take the aggregated features as the query $Q$ and the instance-aware features as the key $K$ and value $V$. 
Finally, we use a 2D convolution layer to convert the refined BEV feature back to 3D shape and make predictions.
% The fusion could be represented as follows:
% \begin{equation}
% \hat{Q}=\operatorname{SoftMax}(Q K^T) V+Q
% \end{equation}

% \subsection{Training Loss}
% Following previous works~\cite{cao2022monoscene, yu2024context}, We employ $\mathcal{L}_{ce}$, $\mathcal{L}^{sem}_{scal}$, and $\mathcal{L}^{geo}_{scal}$ for semantic and geometric occupancy prediction, while $\mathcal{L}_d$ supervises the intermediate depth distribution for view transformation. Moreover, To facilitate faster convergence and enhance the quality of the BEV representation generated by slots, we incorporate the reconstruction loss  $\mathcal{L}_{recon}$. The overall loss function is formulated as follows:
% \begin{align}
%     \mathcal{L} = \lambda_d \mathcal{L}_d + \lambda_r\mathcal{L}_{recon}+ \mathcal{L}_{ce} + \mathcal{L}^{geo}_{scal} + \mathcal{L}^{sem}_{scal}
% \end{align}
% where the loss weights  $\lambda_d$ and $\lambda_r$ are set to 0.001 and 0.1, respectively.

\begin{table*}[ht]
\centering
\setlength{\tabcolsep}{2.1pt}
{
\begin{tabular}{l|c c|ccccccccccccccccccccc}
    \toprule[.05cm]
    \textbf{Method}                               &
    \multicolumn{1}{c}{\textbf{IoU}}             &
    \textbf{mIoU}                                 &
    \rotatebox{90}{\textbf{road}}                 &
    \rotatebox{90}{\textbf{sidewalk}}                 &
    \rotatebox{90}{\textbf{parking}}                 &
    \rotatebox{90}{\textbf{other-grnd.}}                 &
    \rotatebox{90}{\textbf{building}}                 &
    \rotatebox{90}{\textbf{car}}                 &
    \rotatebox{90}{\textbf{truck}}                 &
    \rotatebox{90}{\textbf{bicycle}}                 &
    \rotatebox{90}{\textbf{motorcycle}}                 &
    \rotatebox{90}{\textbf{other-veh.}}                 &
    \rotatebox{90}{\textbf{vegetation}}                 &
    \rotatebox{90}{\textbf{trunk}}                 &
    \rotatebox{90}{\textbf{terrain}}                 &
    \rotatebox{90}{\textbf{person}}                 &
    \rotatebox{90}{\textbf{bicyclist}}                 &
    \rotatebox{90}{\textbf{motorcyclist}}                 &
    \rotatebox{90}{\textbf{fence}}                 &
    \rotatebox{90}{\textbf{pole}}                 &
    \rotatebox{90}{\textbf{traffic-sign}}                 &
    
    % \clsname{road}{road}                 &
    % \clsname{sidewalk}{sidewalk}         &
    % \clsname{parking}{parking}           &
    % \clsname{other-grnd.}{otherground}   &
    % \clsname{building}{building}         &
    % \clsname{car}{car}                   &
    % \clsname{truck}{truck}               &
    % \clsname{bicycle}{bicycle}           &
    % \clsname{motorcycle}{motorcycle}     &
    % \clsname{other-veh.}{othervehicle}   &
    % \clsname{vegetation}{vegetation}     &
    % \clsname{trunk}{trunk}               &
    % \clsname{terrain}{terrain}           &
    % \clsname{person}{person}             &
    % \clsname{bicyclist}{bicyclist}       &
    % \clsname{motorcyclist}{motorcyclist} &
    % \clsname{fence}{fence}               &
    % \clsname{pole}{pole}                 &
    % \clsname{traf.-sign}{trafficsign}
    \\
    % \midrule
    % \multicolumn{22}{l}{\textit{LiDAR-based methods}} \\
    \midrule
    MonoScene   & 34.16          & 11.08          & 54.7          & 27.1          & 24.8          & 5.7           & 14.4          & 18.8          & 3.3          & 0.5          & 0.7          & 4.4          & 14.9          & 2.4           & 19.5          & 1.0          & 1.4          &0.4          & 11.1          & 3.3          & 2.1          \\
    TPVFormer     & 34.25          & 11.26          & 55.1          & 27.2          & 27.4          & 6.5          & 14.8          & 19.2          & 3.7 & 1.0          & 0.5          & 2.3          & 13.9          & 2.6           & 20.4          & 1.1          & 2.4          & 0.3          & 11.0          & 2.9          & 1.5          \\
    VoxFormer    & 42.95 & 12.20          & 53.9          & 25.3          & 21.1          & 5.6           & 19.8          & 20.8          & 3.5          & 1.0          & 0.7          & 3.7          & 22.4          & 7.5           & 21.3          & 1.4          & 2.6 & 0.2          & 11.1          & 5.1          & 4.9          \\
    OccFormer  & 34.53          & 12.32          & 55.9          & 30.3 & 31.5 & 6.5    & 15.7          & 21.6     & 1.2   & 1.5   &1.7    & 3.2          & 16.8    & 3.9    & 21.3    & 2.2   & 1.1   & 0.2          & 11.9    & 3.8   & 3.7  \\
    MonoOcc &- & 13.80 &55.2 &27.8 &25.1 & 9.7  &21.4 &23.2 & \underline{5.2}  & 2.2 &1.5 &5.4 &24.0 &8.7 &23.0 &1.7 &2.0 &0.2 &13.4 &5.8 & 6.4  \\
    Symphonies    & 42.19          &15.04 & 58.4 & 29.3  & 26.9  & 11.7 &24.7 & 23.6 & 3.2          & 3.6 & \textbf{2.6} & \underline{5.6} & 24.2 & 10.0 &23.1 & \textbf{3.2} & 1.9          & \textbf{2.0} &16.1&7.7 & 8.0 \\ 
    LOMA & 43.01       & 15.10 & 58.0 & 31.8   & 32.2   & 9.5 &25.3 & 24.9 & 4.1  & 1.7 & 1.7 & \textbf{6.4} & \underline{25.6} &8.7 & 24.7 & 1.4&1.7 & 0.6 & 16.8 & 6.5  &6.1 \\
    CGFormer  & \underline{44.41}          & 16.63         & 64.3         & 34.2 & \textbf{34.1} & 12.1   & \underline{25.8}       & 26.1     & 4.3   & \underline{3.7}  & 1.3  & 2.7         & 24.5    & \underline{11.2}    & 29.3 & 1.7   & \textbf{3.6}   & 0.4        & 18.7   & \underline{8.7}   & \underline{9.3}  \\
    % L2COcc-C   &CVPR'25  & 44.31          & 17.03          & 66.00     & 35.00 & 33.10 & 13.50   & 25.10          & \textbf{27.20}     & 3.00   & 3.50   & 3.60   & 4.30          & 25.20    & 11.50    & 30.10  & 1.50   & 2.40   & 0.20          & 20.50    & 9.10   & 8.90  \\
    % ScanSSC   &CVPR'25  & \underline{44.54}          & \textbf{17.40}          & \textbf{66.20}          & \textbf{35.90} & \textbf{35.10} & \underline{12.50}   & \underline{25.30}          & \textbf{27.10}     & 3.50   & 3.50   & \textbf{3.20}   &\underline{6.10}         & 25.20    & 11.00    & \underline{30.60}  &1.80   & \textbf{5.30}   & 0.70          & \underline{20.50}    & \underline{8.40}   &  8.90   \\
    HTCL & 44.23         & \underline{17.09}        & \underline{64.4}          & \underline{34.8}  & \underline{33.8} & \underline{12.4}  & \textbf{25.9}         & \textbf{27.3}    & \textbf{5.7}   & 1.8  & \underline{2.2}  &5.4         & 25.3    & 10.8    & \textbf{31.2}  &1.1  & \underline{3.1} & 0.9         & \textbf{21.1}  & \textbf{9.0}   &  8.3  \\
    Ocean (Ours)  & \textbf{45.62}  & \textbf{17.40} & \textbf{65.1} & \textbf{34.9}   & 33.7   & \textbf{12.8} & 25.4 &\underline{26.8} & \underline{5.2}  & \textbf{4.9}& 1.9 & \underline{5.6}  & \textbf{26.9} &\textbf{11.6} & \underline{30.7} & \underline{2.3} & 2.2 & \underline{1.7} &\underline{20.8} &\underline{8.7} &\textbf{9.5} \\
    \bottomrule[.05cm]
\end{tabular}
}
\caption{Quantitative results on SemanticKITTI test set. The best and second results are in \textbf{bold} and \underline{underlined}, respectively.}
\label{tab:sem_kitti_test}
\end{table*}


\begin{table*}[ht]
\centering
% \newcommand{\clsname}[2]{
%     \rotatebox{90}{
%         \hspace{-5pt}
%         \textcolor{#2}{$\blacksquare$}
%         \hspace{-5pt}
%         % \renewcommand\arraystretch{0.6}
%         \begin{tabular}{l}
%             #1                                      \\
%             % \hspace{-4pt} ~\tiny(\semkitfreq{#2}\%) \\
%         \end{tabular}
%     }}
\setlength{\tabcolsep}{2.1pt}
% \scalebox{0.8}
{
\begin{tabular}{l|c c|ccccccccccccccccccccc}
        \toprule[.05cm]
        \textbf{Method}                               &
        % IoU &
        % mIoU &
        \multicolumn{1}{c}{\textbf{IoU}}             &
        \textbf{mIoU}                                               &
        \rotatebox{90}{\textbf{car}}                 &
        \rotatebox{90}{\textbf{bicycle}}                 &
        \rotatebox{90}{\textbf{motorcycle}}                 &
        \rotatebox{90}{\textbf{truck}}                 &
        \rotatebox{90}{\textbf{other-veh.}}                 &
        \rotatebox{90}{\textbf{person}}                 &
        \rotatebox{90}{\textbf{road}}                 &
        \rotatebox{90}{\textbf{parking}}                 &
        \rotatebox{90}{\textbf{sidewalk}}                 &
        \rotatebox{90}{\textbf{other-grnd.}}                 &
        \rotatebox{90}{\textbf{building}}                 &
        \rotatebox{90}{\textbf{fence}}                 &
        \rotatebox{90}{\textbf{vegetation}}                 &
        \rotatebox{90}{\textbf{terrain}}                 &
        \rotatebox{90}{\textbf{pole}}                 &
        \rotatebox{90}{\textbf{traffic-sign}}                 &
        \rotatebox{90}{\textbf{other-struct.}}                 &
        \rotatebox{90}{\textbf{other-obj.}}                 &

        % \multicolumn{1}{c}{\clsname{car}{car}}                      &
        % \multicolumn{1}{c}{\clsname{bicycle}{bicycle}}              &
        % \multicolumn{1}{c}{\clsname{motorcycle}{motorcycle}}        &
        % \multicolumn{1}{c}{\clsname{truck}{truck}}                  &
        % \multicolumn{1}{c}{\clsname{other-veh.}{othervehicle}}      &
        % \multicolumn{1}{c}{\clsname{person}{person}}                &
        % \multicolumn{1}{c}{\clsname{road}{road}}                    &
        % \multicolumn{1}{c}{\clsname{parking}{parking}}              &
        % \multicolumn{1}{c}{\clsname{sidewalk}{sidewalk}}            &
        % \multicolumn{1}{c}{\clsname{other-grnd.}{otherground}}      &
        % \multicolumn{1}{c}{\clsname{building}{building}}            &
        % \multicolumn{1}{c}{\clsname{fence}{fence}}                  &
        % \multicolumn{1}{c}{\clsname{vegetation}{vegetation}}        &
        % \multicolumn{1}{c}{\clsname{terrain}{terrain}}              &
        % \multicolumn{1}{c}{\clsname{pole}{pole}}                    &
        % \multicolumn{1}{c}{\clsname{traf.-sign}{trafficsign}}       &
        % \multicolumn{1}{c}{\clsname{other-struct.}{otherstructure}} &
        % \multicolumn{1}{c}{\clsname{other-obj.}{otherobject}}
        \\ 
        \midrule
        % \multicolumn{21}{l}{\textit{LiDAR-based methods}} \\
        % \hline
        MonoScene   & 37.87          & 12.31        & 19.3        & 0.4        & 0.6        & 8.0         & 2.0         & 0.9        & 48.4        & 11.4        & 28.1        & 3.3        & 32.9        & 3.5        & 26.2        & 16.8        & 6.9         & 5.7        & 4.2         & 3.1         \\
        TPVFormer          & 40.22       & 13.64        & 21.6        & 1.1        & 1.4        & 8.2         & 2.6         & 2.4        & 53.0       & 12.0        & 31.1        & 3.8        & 34.8        & 4.8        & 30.1        & 17.5        & 7.5         & 5.9        & 5.5         & 2.7         \\
        VoxFormer           & 38.76         & 11.91        & 17.8        & 1.2        & 0.9        & 4.6         & 2.1         & 1.6        & 47.0        & 9.7         & 27.2        & 2.9        & 31.2        & 5.0        & 29.0        & 14.7        & 6.5         & 6.9        & 3.8         & 2.4         \\
        OccFormer           & 40.27      & 13.81        & 22.6        & 0.7        & 0.3        & 9.9         & 3.8         & 2.8        & 54.3        & 13.4        & 31.5        & 3.6        &  36.4 & 4.8        & 31.0        &  19.5 & 7.8         & 8.5        & 7.0         & 4.6         \\
        Symphonies        & 44.12    & 18.58 & \textbf{30.0} & 1.9  & \textbf{5.9} & \textbf{25.1} & \textbf{12.1} & \textbf{8.2} &  54.9  & 13.8 &  32.8 & \textbf{6.9} & 35.1        &  \underline{8.6}  &  38.3  & 11.5        & 14.0 & 9.6  & \textbf{14.4} & \textbf{11.3} \\  
     
        CGFormer     & \underline{48.07}    & \underline{20.05} & \underline{29.9} & \underline{3.4} &4.0  & \underline{17.6} & 6.8 & 6.7 &\textbf{63.9} &\textbf{17.2}& \underline{40.7} &\underline{5.5} & 42.7 &8.2 &38.8 &\textbf{24.9} &\underline{16.2} &\underline{17.5} &10.2 &6.8 \\
        % ScanSSC     & \textbf{48.29}    & \underline{20.14} & 29.91 & \textbf{3.78} & 4.28 & 14.34  &\underline{9.08}& 6.65 &62.21 &\textbf{18.16}& 40.19 &\underline{5.16} &42.68 &\underline{8.83} &\underline{38.84} &\textbf{25.50} &\underline{16.60}&\underline{19.14}&10.30 &6.89 \\
        SGFormer        & 46.35    & 18.30 & 27.8  & 0.9 & 2.6 & 10.7 & 5.7 & 4.3 &  61.0  & 13.2 &  37.0 & 5.1 & \underline{43.1}   &  7.5  &  \underline{39.0} & \textbf{24.9}       & 15.8 & 16.9  & 8.9 & 5.3 \\  
        
        Ocean (Ours)   & \textbf{48.19}   & \textbf{20.28} & 29.3 & \textbf{3.7} & \underline{4.6} & 15.1 & \underline{7.7} & \underline{6.8} & \underline{63.7} &  \underline{17.0}  & \textbf{40.9} & 5.0 &\textbf{43.7} & \textbf{8.9} & \textbf{39.2}    &  \underline{24.7} & \textbf{16.7} & \textbf{19.2}     & \underline{10.8} & \underline{8.3} &  \\
        \bottomrule[.05cm]
        \end{tabular}
}
\caption{Quantitative results on SSCBench-KITTI360 test set.  The best and second results are in \textbf{bold} and \underline{underlined}.}

\label{tab:kitti_360_test}
\end{table*}


\section{Experiments}
% In this section, we compare our method on SemanticKITTI~\cite{behley2019semantickitti} and SSCBench-KITTI-360~\cite{li2023sscbench} datasets with previous methods. The specific comparative experiment details are in Sec.~\ref{Expriments:main results}. Meanwhile, We perform ablation experiments in Sec.~\ref{Expriments:ablation study} to fully understand the efficiency of our network. Moreover, we visualize the results and achieve qualitative comparison in Sec.~\ref{Expriments:visualization}.


% \subsection{Experimental Setup}
% \textbf{Dataset.} 
% SemanticKITTI~\cite{behley2019semantickitti} uses 22 sequences with a $256 \times 256 \times 32$ voxel grid and 20 classes.
% SSCBench-KITTI360~\cite{li2023sscbench} includes 9 sequences and 19 classes, with the same voxel settings.
%SemanticKITTI consists of $22$ driving sequences, divided into $10/1/11$ sequences for training/validation/testing. 
% % These sequences capture diverse environments, including inner-city traffic, residential areas, highways, and rural roads. 
% The dataset focuses on a 3D space of $51.2$ meters in front of the vehicle, $25.6$ meters to each side, and $6.4$ meters in height, with a voxel resolution of $0.2m$. This setup produces a prediction volume of $256 \times 256 \times 32$ voxels, with each voxel labeled into one of $20$ classes.
% % ($1$ free and $19$ semantic classes).
% %image size %
% SSCBench-KITTI360 primarily consists of suburban scenes and includes $7$ sequences for training, $1$ for validation, and $1$ for testing, covering a total of $19$ semantic classes. The other settings are the same with SemanticKITTI.

%\textbf{Metric.} We report the intersection over union (IoU) and mean IoU (mIoU) metrics in the evaluation, which measure the geometric prediction of each voxel and semantic prediction of occupied voxels, respectively.
% IoU emphasizes spatial geometry, identifying whether each voxel in a 3D space is occupied or empty. In contrast, mIoU combines both semantic and geometric information and classifies the contents of each occupied voxel. 
% offering a more comprehensive understanding of the scene by classifying the contents of each occupied voxel.

\subsection{Implementation Details}
%1.骨干网络   2.将图像变为(1280,384),3. 网络层数  4. Adam lr 训练次数 decay rate 5. 3090 %
Following previous works~\cite{cao2022monoscene}, we use EfficientNetB7~\cite{tan2019efficientnet} as our 2D backbone.
%Multi-scale features and depth distributions are predicted by lightweight and independent networks.
The SemGroup Dual Attention block consists of $3$ layers.
The query proposals from SGDA are scattered back to a 3D feature of size $128 \times 128 \times 16$ with $128$ channels, which is then processed by the ILD. 
We train Ocean on $4$ GeForce 3090 GPUs, using the AdamW optimizer~\cite{loshchilov2017decoupled} with an initial learning rate of 3e-4 and weight decay of 0.01.

\begin{figure*}[t]
    \centering
    \includegraphics[width=0.9\linewidth]{imgs/main_vis.pdf}
    \caption{Qualitative visualizations on SemanticKITTI validation set. %Compared to previous methods, our approach shows superior and more distinct predictions, particularly in distinguishing individual instances in complex scenes.
    }
    \label{fig:viz}
\end{figure*}



\subsection{Comparisons with the State-of-the-Art Methods} \label{Expriments:main results}
% This result demonstrates that aggregating features at the instance level feature in 3D space can effectively 
%demonstrating the effectiveness of the 3D object-centric learning paradigm.
%1. baseline 两个数据集, 2. % 加上可视化
% 如表1 表2,我们的方法在两个数据集上展现了最优异的结果
As shown in Table~\ref{tab:sem_kitti_test} and Table~\ref{tab:kitti_360_test}, our method achieves the best performance on both the SemanticKITTI~\cite{behley2019semantickitti} and the SSCBench-KITTI360~\cite{li2023sscbench} datasets.  
On the SemanticKITTI dataset, Ocean reaches an IoU of 46.40 and a mIoU of 17.39, outperforming CGFormer by 1.21 and 0.77, respectively. It also surpasses HTCL~\cite{li2024hierarchical}, which utilizes temporal information, demonstrating the effectiveness of our instance-level feature aggregation. 
% 我们的方法能够更优异的判别出场景中边缘或轮廓信息明显的前景和背景,比如图中的汽车和路面
Additionally, the SemanticKITTI validation visualizations in Figure~\ref{fig:viz} clearly demonstrate that our method achieves superior capability in capturing fine-grained scene structure, especially for objects with strong edge or contour cues, such as cars and roads.
On the SSCBench-KITTI360 dataset, Ocean exceeds CGFormer by 0.12 in IoU and 0.23 in mIoU, and further outperforms SGFormer~\cite{guo2025sgformer}, which leverages additional remote sensing modalities. These results confirm the advantages of our object-centric 3D scene understanding approach.
% 特别的是, 在两个数据集上,Ocean可以以最优异的结果识别出表面一致的物体,比如地面,sidewalk, 甚至细小的物体,如杆、交通信号, 也能取得优异的结果, 这些得以于先验信息和SGDA3D算法的设计, 对于表面信息繁杂的物体, 分割算法难以分出的物体,比如bicycle和vegetation,Ocean也能取得SOTA的结果,这是因为Ocean通过GSGA算法使模型不严格依赖先验信息,弥补了它的信息丢失以及分割错误的问题。
On both datasets, Ocean achieves outstanding performance in recognizing objects with consistent surface information, such as the road, sidewalks, and even smaller items such as poles and traffic signals, thanks to the guidance of prior knowledge and the design of the SGA3D algorithm. For objects with more complex surface information, like bicycles and vegetation, which are difficult for segmentation algorithms to distinguish, Ocean also delivers superior results, as the GSGA algorithm reduces reliance on prior information, mitigating information loss and segmentation errors.

\begin{table}[t]
\centering
% \setlength{\tabcolsep}{5.0pt}
\setlength{\tabcolsep}{1mm}
{
\begin{tabular}{c|cc|cc|cc}
\toprule[.05cm]
\multirow{2}{*}{}  & \multicolumn{2}{c|}{\textbf{SGDA}} & \multicolumn{2}{c|}{\textbf{ILD}} & \multicolumn{2}{c}{\textbf{Performance}} \\ 
\cline{2-3} \cline{4-5} \cline{6-7}
& \textbf{SGA3D}  & \textbf{GSGA}      & \textbf{LA}         & \textbf{DID}        & \textbf{IoU ↑}         & \textbf{mIoU ↑} \\
\midrule
   
% M1 &     &     &   &      &   15.14     & 44.85        \\
M0 &     &     &        &      & 44.62  &   15.80       \\
% M2 &   \checkmark   &      &  & &  15.59        &   44.74            \\
M1 &   \checkmark  &     &   &       &   45.77    &  16.49         \\
% M3 &   \checkmark   &  \checkmark  &  &  &   15.85     & 44.89    \\      
M2    &  \checkmark  & \checkmark  & &  &  46.01  &   16.80   \\  

M3  &   \checkmark           &  \checkmark   &  \checkmark&   &  45.39 &  17.10   \\   
% M4 &     &              &  \checkmark   &    &  15.42  &  45.41     \\   
% M5 &   \checkmark &              &  \checkmark    &             &                &         \\      
   
M4  &  \checkmark & \checkmark & \checkmark & \checkmark        &   \textbf{46.40} & \textbf{17.39} \\ 
\bottomrule[.05cm]
\end{tabular}}
\caption{Ablation study on each module in Ocean.}
\label{tab:main_module}
\end{table}

\subsection{Ablation study} \label{Expriments:ablation study}
To analyze the effectiveness of each module, we conduct ablation studies on the SemanticKITTI validation set.
% To comprehensively evaluate the effectiveness and contribution of each component in our framework, we conduct detailed ablation studies on the SemanticKITTI validation set.

\noindent\textbf{Main Modules.} Table~\ref{tab:main_module} presents an ablation study comparing the core modules of our method. 
Compared to our baseline, the introduction of prior knowledge and the application of SGA3D result in an improvement of 1.15/0.69 in IoU/mIoU for M1, demonstrating the effectiveness of our design. However, due to the mistakes and omissions in the prior masks, the model has not fully learned the scene information. Therefore, M2 further incorporates GSGA and achieves 46.01/16.80 in IoU/mIoU. This result demonstrates that aggregating features at the instance level feature in 3D space can effectively enhance both semantic and geometric perception.
Furthermore, after incorporating local feature diffusion, mIoU improved by 0.3. However, due to the voxel2bev operation, which enhances the instance's semantic information, some geometric information is inevitably lost. 
To address this issue, we further incorporate the Dynamic Instance Decoder (DID), which adopts a generative paradigm to enhance the overall scene semantic information and partially mitigate the loss of geometric details. 




\begin{table}[t]
\centering
% \setlength{\tabcolsep}{5.0pt}
\setlength{\tabcolsep}{1mm}
% \renewcommand\arraystretch{1.0}
{
\begin{tabular}{c|cccc}
\toprule[.05cm]
\textbf{Method} & \textbf{IoU ↑} & \textbf{mIoU ↑} & \textbf{Param} & \textbf{FLOPs}  \\
\midrule
SGA3D(Ours)   & 46.40   & \textbf{17.39} &  0.280M  & 1.333G \\
SGA3D w/o Ms.  & \textbf{46.44} & 16.64  & 0.280M  & 1.298G  \\
SGA3D w/o Ext.  & 45.84 & 17.30  & 0.280M  & 1.333G  \\
DFA2D  &  46.14 & 16.39  & 0.202M & 2.231G  \\
DFA3D  &  46.15 & 16.97  & 0.330M & 2.807G  \\
\bottomrule[.05cm]
\end{tabular}
}
\caption{Ablation study on 3D Semantic Group Attention.}
\label{tab:sga}
\end{table}

% \begin{table}[t]
% \centering
% \renewcommand\arraystretch{0.9}
% \begin{tabular}{cc|ccc}
% \toprule[.05cm]
% \multicolumn{2}{c|}{Method}  & IoU ↑ & mIoU ↑ & FLOPs ↓ \\
% \midrule

% F1 & SGA3D (Full Model)    & \textbf{46.40} & \textbf{17.39}  & 1.332G \\ % 0.280M

% F2 &w / o Multi-Scale        & 46.44 & 16.64 & 1.332G \\  %0.280
% F3 &w / o 3D Extension      & 46.38 & 17.45  & 1.338G \\
% \bottomrule[.05cm]
% \end{tabular}
% \caption{\textbf{Ablation study in 3D Semantic Group Attention.}}
% \label{tab:local_fusion}
% \end{table}

\begin{table}[t]
\centering
% \setlength{\tabcolsep}{9.0pt}
\setlength{\tabcolsep}{1mm}
% \renewcommand\arraystretch{0.9}
\begin{tabular}{c|cc}
\toprule[.05cm]
\textbf{Method}  & \textbf{IoU ↑} & \textbf{mIoU ↑} \\
\midrule

Full Model(Ours)    & \textbf{46.40} & \textbf{17.39}\\ 
w/o GSGA        & 46.13 & 16.98 \\
GSGA w/o Sim. & 46.07 & 16.36 \\

\bottomrule[.05cm]
\end{tabular}
\caption{Ablation study in Global Similarity-Guided Attention.}
\label{tab:SGDFA}
\end{table}


\begin{table}[t]
\centering
% \setlength{\tabcolsep}{1mm}
% \setlength{\tabcolsep}{5.0pt}
% \renewcommand\arraystretch{0.8}
{
\begin{tabular}{c|cc}
\toprule[.05cm]
\textbf{Method} & \textbf{IoU ↑} & \textbf{mIoU ↑} \\
\midrule
 Direct Sum  & 45.79 &   16.45  \\
  Softmax-Weighted    & 45.89 & 16.88  \\
 Sigmoid-Weighted    & 46.30 & 16.92    \\
 Dynamic Selection (Ours)   &\textbf{46.40}     & \textbf{17.39}    \\ 
\bottomrule[.05cm]
\end{tabular}
}
\caption{Ablation study in Dynamic Instance Decoder.}
\label{tab:instance_fusion}
\end{table}

\noindent\textbf{3D Semantic Group Attention.} 
% . 正如表3所示, 我们首先分别去除了SGA3D中两个核心的设计, 去掉多尺度信息之后mIoU下架下降了0.75, 这说明了多尺度信息对于语义分割的重要性, 不同尺度的mask所看见的实例也不同.
%去掉3D Extension后, IoU和mIoU下降了0.17/1.02, 这说明了2D mask会限制模型对3D世界的感知, 同时也证明我们的设计突破了2D mask的限制.
% 不仅如此, 我们还用DFA3D和DFA2D替换了SGA3D, 我们提出的SGA3D用更少的参数量更少的计算量达到了最好的性能. 具体来说, 我们的方法优于DFA2D和DFA3D分别1.00和0.42.
% We replace the proposed SGA3D with Conv, DFA2D~\cite{DeformableDETR} and DFA3D~\cite{DFA3D}. As shown in Table.~\ref{tab:sga}, our proposed SGA3D achieves the best performance with fewer parameters and lower computational costs. Specifically, our method outperforms DFA2D and DFA3D by 0.94 and 0.36 in mIoU, respectively.
% This result verifies that, compared to scene-level aggregation, object-level aggregation can better capture semantic details, leading to better semantic prediction.
% 我们首先验证SGA3D两个核心组成, 在加入多尺度图像特征后,mIoU提升了0.75,证明了多尺度图像特征对语义信息的重要性. 将SGA拓展到SGA3D后, IoU提升了0.56, 表明了3D Extension突破了2D 先验对3D SSC任务的限制.
As shown in Table~\ref{tab:sga}, we first evaluate the two core components of SGA3D. Without(w/o) the multi-scale image features, the model experiences a significant drop in semantic understanding, while removing the 3D extension design leads to a noticeable decline in geometric reasoning, demonstrating the distinct and indispensable contributions of both components to the effectiveness of our method.
Furthermore, substituting SGA3D with DFA3D or DFA2D demonstrates the superior performance of SGA3D, which achieves mIoU gains of 1.00 and 0.42 over DFA2D and DFA3D, respectively, while maintaining a lower parameter count and computational overhead.


\noindent\textbf{Global Similarity-Guided Attention.}
% 我们对GSGA模块进行深度消融实验, 在去除该模块后, IoU和mIoU都降低了, 分别证明了证明了SGDFA能够解决SGA3D依赖前景造成的信息丢失的问题, 这体现了该模块能够突破mask的空间限制. 另外, 简单的全局特征处理, 没有分割模型特征引导,  会对现有特征造成模糊, 导致 IoU和mIoU都有一定幅度的下降, 这证明了我们设计是合理的.
We conducted an ablation study on the GSGA module in Table~\ref{tab:SGDFA}.
Removing GSGA leads to a decline in both IoU and mIoU metrics, indicating that it effectively mitigates the information loss caused by the foreground dependency in SGA3D and alleviates the spatial constraints imposed by the prior mask.
In a separate experiment, removing similarity-guided global feature processing and replacing it with a naive global operation introduces ambiguity into both geometry and semantic features causing a degradation in overall performance.  
These results collectively validate the effectiveness of our design.




\noindent\textbf{Dynamic Instance Decoder.}
In Table~\ref{tab:instance_fusion}, we compare different fusion strategies in the Dynamic Instance Decoder. Compared to directly summing, the weighted approaches including sigmoid and softmax show better performance, verifying that different instances have different effects in fusion. Finally, by dynamically predicting the decision of each instance, our approach achieves the best performance.


% Moreover, we explore the effectiveness of the instance feature in the Dynamic Instance Decoder in Table.~\ref{tab:instance_fusion}. The comparison between I1 and I2 shows that the image has better semantic information while the voxel has better geometry information, which is also consistent with our hypothesis. Therefore, by utilizing both of them to provide instance information, our method achieves the best performance.



% \noindent\textbf{BEV Refinement.}
% In Table~\ref{tab:bevrefin}, we compare BEV feature refinement strategies, including convolution, deformable attention, and vanilla attention. Local operations outperform global vanilla attention. While convolution achieves the highest IoU, our method delivers the best mIoU with fewer parameters and lower computational cost, striking a better balance between accuracy and efficiency.


% \begin{table}[t]
% \centering
% \renewcommand{\tabcolsep}{2 pt}
% \renewcommand\arraystretch{0.8}
% {
% \begin{tabular}{c|ccrr}
% \toprule[.05cm]
% Local Size  & IoU ↑ &mIoU ↑  &Param & FLOPs \\
% \midrule
% 4  & 16.17  &45.36  &0.330M &5.457G \\
% 8   & \textbf{16.40}  &45.50   &  0.332M  & 5.658G   \\
% 16  & 15.45 & \textbf{45.57} & 0.338M & 6.463G \\
% 32  & 13.18 &  45.13 &0.362M &9.685G  \\
% % W5& 64  &  &   &0.459M &22.57G  \\
% \bottomrule[.05cm]
% \end{tabular}
% }
% \caption{\textbf{Ablation study on Local Size.}}
% \label{tab:local_size}
% \end{table}

% \begin{table}[t]
% \centering
% \setlength{\tabcolsep}{4.0pt}
% % \renewcommand\arraystretch{0.8}
% {
% \begin{tabular}{c|ccrr}
% \toprule[.05cm]
% Method  & IoU ↑ &mIoU ↑  &Param & FLOPs \\
% \midrule
% Attention   & 46.14  & 16.36   &  0.188M & 11.64G    \\
% DFA2D & 46.55 & 16.81 & 0.124M &2.034G  \\
% Conv  & \textbf{46.54}  &17.00  &0.378M & 6.191G \\
% Ours  & 46.40 &  \textbf{17.39} &0.332M &5.658G  \\
% % W5& 64  &  &   &0.459M &22.57G  \\
% \bottomrule[.05cm]
% \end{tabular}
% }
% \caption{\textbf{Ablation study on BEV Refinement.}}
% \label{tab:bevrefin}
% \end{table}

% \begin{table}[t]
% \centering
% % \renewcommand{\tabcolsep}{2.0pt}
% \renewcommand\arraystretch{0.8}
% {
% \begin{tabular}{c|ccc}
% \toprule[.05cm]
% Method  & IoU ↑ & mIoU ↑  &Latency ↓ \\
% \midrule
% CGFormer & 45.99 & 16.89 & 205 \\ % &122 
% SAM-based Ocean  & 16.17  &45.36  & 780 \\
% RepvitSAM-based Ocean   & \textbf{16.40}  &45.50   &  220   \\
% MobileSAM-based ocean  & 46.40 & \textbf{17.39} & 230 \\
% % W5& 64  &  &   &0.459M &22.57G  \\
% \bottomrule[.05cm]
% \end{tabular}
% }
% \caption{\textbf{Ablation study on different segment methods.}}
% \label{tab:dsm}
% \end{table}

% \subsection{Visualization}\label{Expriments:visualization}
% As illustrated in Figure.~\ref{fig:viz}, Ocean shows superior performance compared to prior methods~\cite{cao2022monoscene, li2023voxformer, zhang2023occformer, jiang2024symphonize} on the SemanticKITTI validation set. 
% Specifically, Ocean can distinguish individual cars in different scenes, even if they are located close to each other or are partially occluded due to being far away from the observer's perspective.
% Specifically, Ocean clearly and effectively distinguishes individual cars, even when they are located in close proximity along the roadside.
% Specifically, Ocean can clearly distinguish each car in a densely packed cluster, whereas previous methods, such as OccFormer, tend to represent them as a single or a few elongated vehicles.


\section{Conclusion}
We propose Ocean, an object-centric framework for vision-based 3D Semantic Scene Completion. Unlike ego-centric approaches, Ocean models scenes from an instance-level perspective to enhance semantic and geometric understanding. It leverages instance priors to guide object-centric feature learning through attention-based aggregation and spatial diffusion.
Ocean achieves state-of-the-art performance on the SemanticKITTI and SSCBench-KITTI-360 datasets, showing the effectiveness of the object-centric paradigm.

\section{Acknowledgments}
This work was supported in part by the National Natural Science Foundation of China under Grants 62073066 and in part by 111 Project under Grant B16009.

% \bigskip
% \noindent Thank you for reading these instructions carefully. We look forward to receiving your electronic files!

\bibliography{aaai2026}

%%%%%%%%%%%%%%%%%%%%%%%%%%%%%%
\clearpage
\title{\vspace{-1cm}Supplementary Material for Ocean\vspace{0.2 cm}}



\section{A\hspace{1em}More Implementation Details}
\subsection{Dataset}
SemanticKITTI consists of $22$ driving sequences, divided into $10/1/11$ sequences for training/validation/testing. 
% These sequences capture diverse environments, including inner-city traffic, residential areas, highways, and rural roads. 
The dataset focuses on a 3D space of $51.2$ meters in front of the vehicle, $25.6$ meters to each side, and $6.4$ meters in height, with a voxel resolution of $0.2m$. This setup produces a prediction volume of $256 \times 256 \times 32$ voxels, with each voxel labeled into one of $20$ classes.
% ($1$ free and $19$ semantic classes).
%image size %
SSCBench-KITTI360 primarily consists of suburban scenes and includes $7$ sequences for training, $1$ for validation, and $1$ for testing, covering a total of $19$ semantic classes. The other settings are the same as SemanticKITTI.
\subsection{Metrics}
We report the intersection over union (IoU) and mean IoU (mIoU) metrics in the evaluation, which measure the geometric prediction of each voxel and semantic prediction of occupied voxels, respectively.
IoU emphasizes spatial geometry, identifying whether each voxel in a 3D space is occupied or empty. In contrast, mIoU combines both semantic and geometric information and classifies the contents of each occupied voxel. 
% offering a more comprehensive understanding of the scene by classifying the contents of each occupied voxel.

\subsection{Training Loss}
Following previous work, We employ $\mathcal{L}_{ce}$, $\mathcal{L}^{sem}_{scal}$, and $\mathcal{L}^{geo}_{scal}$ for semantic and geometric occupancy prediction, while $\mathcal{L}_d$ supervises the intermediate depth distribution for view transformation. Moreover, To facilitate faster convergence and enhance the quality of the BEV representation generated by slots, we incorporate the reconstruction loss  $\mathcal{L}_{recon}$. The overall loss function is formulated as follows:
\begin{align}
    \mathcal{L} = \lambda_d \mathcal{L}_d + \lambda_r\mathcal{L}_{recon}+ \mathcal{L}_{ce} + \mathcal{L}^{geo}_{scal} + \mathcal{L}^{sem}_{scal}
\end{align}
where the loss weights  $\lambda_d$ and $\lambda_r$ are set to 0.001 and 0.1, respectively.




\begin{figure*}[t]
    \centering
    \includegraphics[width= 0.8 \linewidth]{imgs/sup_failure.pdf}
    \caption{Failure case visualization. Insufficient visual information caused by long-range distance or severe occlusion poses a major challenge for object recognition. Under such conditions, the model often struggles to extract discriminative features, which leads to reduced recognition accuracy and increased ambiguity in classification.
    }
    \label{fig:fail}
\end{figure*}


\section{B\hspace{1em}Additional Experimental Results}
\subsection{More Exploration of Integrating MobileSAM}
In addition to SGA3D, we investigate an alternative strategy for integrating MobileSAM into the SSC task. Specifically, after the semantic grouping process, we pool the pixel features within each cluster to generate an instance feature $\mathcal{U} \in \mathbb{R}^{1\times C_2}$. This instance feature is then used as the key $K$ and value $V$ in a cross-attention mechanism alongside the proposal features $\mathcal{Q}$. We refer to this method as Semantic Instance Attention (SIA). 
Furthermore, based on the clustered features, we perform a simple feature fusion between the proposal features and the corresponding image features of the same instance, a process we refer to as Semantic Group Fusion (SGF). It is important to note that we focus solely on SGA3D in this comparison, while the remaining modules are kept consistent with our proposed method, Ocean.
We then compare these two variants with our full method and an ablated version without the SGA3D module to evaluate their relative performance.


As shown in Table~\ref{tab:sia}, SIA experiences a performance drop of 0.62/0.54 in IoU/mIoU due to the direct pooling of instance features, which results in substantial information loss. In comparison, SGF leverages all instance features and facilitates intra-instance feature interaction, thereby achieving relatively better performance. However, this improvement comes at the expense of increased parameter count and computational complexity. In contrast, our method, without relying on depth information, achieves improvements of 0.08/0.10 in IoU/mIoU while maintaining the lowest number of parameters and computational cost. 
Furthermore, with the integration of the 3D Extension, our model reaches 46.40 IoU and 17.39 mIoU. 
These results demonstrate the efficiency of our SGA3D module, which establishes a strong foundation for the object-centric paradigm.

\begin{table}[h]
\centering
\setlength{\tabcolsep}{3 pt}
% \renewcommand\arraystretch{0.8}
{
\begin{tabular}{cc|cccc}
\toprule[.05cm]
\multicolumn{2}{c|}{\textbf{Method}} & \textbf{IoU ↑} & \textbf{mIoU ↑} & \textbf{Param} & \textbf{FLOPs}  \\
\midrule
S1       & SIA     & 45.78 & 16.85 &  0.280M  & 1.333G \\
S2       & SGF  & 45.76 & 17.20   &1.516M & 7.197G   \\
S3       & Ours(w/o Ext.) & 45.84 & 17.30  & 0.280M  & 1.333G \\
S4       & Ours   &\textbf{46.40}     & \textbf{17.39} &  0.280M  & 1.333G   \\ 
% 46.23 16.37
\bottomrule[.05cm]
\end{tabular}
}
\caption{Comparison of Intergating MobileSAM.}
\label{tab:sia}
\end{table}

\subsection{SemanticKITTI Validation Performance}
We show the comparison between our method and the previous method in the SemanticKITTI validation set in Table~\ref{tab:sem_kitti_val}.
Compared to other methods, our approach still achieves the best results in terms of IoU and mIoU.





% \subsection{Voxel and mobileSAM feature similarity visualization}
% 



\section{C\hspace{1em}Failure Case Analysis}
% 我们展示了Ocean生成的一些错误例子在图3中。 我们发现我们提出的方法Ocean对远距离、信息不足的物体十分敏感。第一行在图3中, 由于bicyclist距离较远,虽然我们能够检测到那个位置存在一个物体,但是由于距离过远,算法无法辨认出这个物体是person还是bicyclist, 第二个例子中的汽车也是由于距离过远,且有遮挡,物体信息不足导致算法无法识别。 我们后续会尝试使用多模态加时序的方式来解决这些问题。
We present some representative failure cases of Ocean in Figure~\ref{fig:fail}. 
We observe that Ocean may struggle to recognize certain objects under extremely limited visual conditions, such as long distances or heavy occlusions.
%, which can result in misclassification or complete failure of recognition.
In the first row of Figure~\ref{fig:fail}, although the model successfully detects the presence of an object, it fails to distinguish between a person and a bicyclist due to the large distance. Similarly, in the second row, the vehicle is heavily occluded and located far from the sensor, leading to insufficient information for accurate recognition. In future work, we plan to address these limitations by incorporating multimodal and temporal cues.

\section{D\hspace{1em}Visualization}
\subsection{More visualizations of Ocean}
We present additional qualitative results in Figure~\ref{fig:supp_viz1}, showcasing comparisons of our method with existing approaches on the SemanticKITTI validation set. The visualizations clearly demonstrate that our method excels at capturing fine-grained scene structures and delivers superior, more distinct predictions.

\begin{table*}[ht]
\centering
\newcommand{\clsname}[2]{
    \rotatebox{90}{
        \hspace{-5pt}
        \textcolor{#2}{$\blacksquare$}
        \hspace{-5pt}
        % \renewcommand\arraystretch{0.6}
        \begin{tabular}{l}
            #1                                      \\
            % \hspace{-4pt} ~\tiny(\semkitfreq{#2}\%) \\
        \end{tabular}
    }}
\setlength{\tabcolsep}{2.1pt}
% \scalebox{0.8}
{
\begin{tabular}{l|r r|rrrrrrrrrrrrrrrrrrrrr}
    \toprule[.05cm]
    Method                               &
    % IoU &
    % mIoU &
    \multicolumn{1}{c}{IoU}              &
    mIoU                                 &
    \clsname{road}{road}                 &
    \clsname{sidewalk}{sidewalk}         &
    \clsname{parking}{parking}           &
    \clsname{other-grnd.}{otherground}   &
    \clsname{building}{building}         &
    \clsname{car}{car}                   &
    \clsname{truck}{truck}               &
    \clsname{bicycle}{bicycle}           &
    \clsname{motorcycle}{motorcycle}     &
    \clsname{other-veh.}{othervehicle}   &
    \clsname{vegetation}{vegetation}     &
    \clsname{trunk}{trunk}               &
    \clsname{terrain}{terrain}           &
    \clsname{person}{person}             &
    \clsname{bicyclist}{bicyclist}       &
    \clsname{motorcyclist}{motorcyclist} &
    \clsname{fence}{fence}               &
    \clsname{pole}{pole}                 &
    \clsname{traf.-sign}{trafficsign}
    \\
    % \midrule
    % \multicolumn{22}{l}{\textit{LiDAR-based methods}} \\
    \midrule
    MonoScene   & 36.86          & 11.08          & 56.5          & 26.7          & 14.3          & 0.5           & 14.1          & 23.3          & 7.0           & 0.6          & 0.5          & 1.5           & 17.9          & 2.81           & 29.6          & 1.9          & 1.2          & 0.0          & 5.8          & 4.1          & 2.3          \\
    TPVFormer     & 35.61          & 11.36          & 56.5          & 25.9          & 20.6          & \underline{0.9}           & 13.9          & 23.8          & 8.1           & 0.4         & 0.1          & 4.4          & 16.9          & 2.3           & 30.4          & 0.5          & 0.9          & 0.0          & 5.9          & 3.1          & 1.5          \\
    VoxFormer    & 44.02 & 12.35          & 54.8          & 26.4          & 15.5          & 0.7           & 17.7        & 25.8          & 5.6           & 0.6          & 0.5          & 3.8           & 24.4          & 5.1           & 30.0          & 1.8          & \underline{3.3} & 0.0          & 7.6          & 7.1          & 4.2          \\
    OccFormer  & 36.50          & 13.46          & 58.9          & 26.9          & 19.6          & 0.3           & 14.4          & 25.1          & \textbf{25.5} & 0.8          & 1.2          & 8.5           & 19.6          & 3.9           & 32.6 & 2.8          & 2.8          & 0.0          & 5.6          & 4.3          & 2.9          \\
    Symphonies    & 41.92          & 14.89 & 56.4          & 27.6          & 15.3          & \textbf{1.0}           & 21.6 & 28.7 & 20.4          & 2.5 & \textbf{2.8} & \underline{13.9} & 25.7 & 6.6 & 30.9           &\underline{3.5} & 2.2          & 0.0          & 8.4 & 9.6 & 5.8 \\
    CGFormer  & \underline{45.99}          & 16.87         & \underline{65.5}         & 32.3 & 20.8 & 0.2   & \underline{23.5}       & \textbf{34.3}     & 19.4   & \textbf{4.6}  & \underline{2.7}  & 7.7         & 26.9    & \underline{8.8}    & \underline{39.5} & 2.4   & \textbf{4.1}   & 0.0        & 9.2   & 10.7   & \textbf{7.8}  \\
    % L2COcc-C   &CVPR'25  & 44.31          & 17.03          & 66.00     & 35.00 & 33.10 & 13.50   & 25.10          & \textbf{27.20}     & 3.00   & 3.50   & 3.60   & 4.30          & 25.20    & 11.50    & 30.10  & 1.50   & 2.40   & 0.20          & 20.50    & 9.10   & 8.90  \\
    % ScanSSC   &CVPR'25  & \underline{44.54}          & \textbf{17.40}          & \textbf{66.20}          & \textbf{35.90} & \textbf{35.10} & \underline{12.50}   & \underline{25.30}          & \textbf{27.10}     & 3.50   & 3.50   & \textbf{3.20}   &\underline{6.10}         & 25.20    & 11.00    & \underline{30.60}  &1.80   & \textbf{5.30}   & 0.70          & \underline{20.50}    & \underline{8.40}   &  8.90   \\
    HTCL & 45.51         & \underline{17.13}        & 63.7          & \underline{32.5}  & \textbf{23.3} & 0.1  & \textbf{24.1}         & \textbf{34.3}    & \underline{20.7}   & \underline{4.0}  & \textbf{2.8}  & 12.0         & \underline{27.0}    & \underline{8.8}    & 37.7  & 2.6  & 2.3 & 0.0         & \textbf{11.2}  & \textbf{11.5}   &  \underline{7.0}  \\
    Ocean (Ours)  & \textbf{46.40}  & \textbf{17.39} & \textbf{66.1} & \textbf{34.3}   & \underline{21.9}   & 0.1 & 23.4 &\underline{34.0} & 19.3  & 3.2& 2.0 & \textbf{15.3}  & \textbf{28.1} &\textbf{8.9} & \textbf{40.0} & \textbf{3.7} & 1.2 & \underline{0.0} &\underline{10.3} &\underline{10.8} &\textbf{7.8} \\
    \bottomrule[.05cm]
\end{tabular}
}
\caption{Quantitative results on SemanticKITTI validation set. The best and second results are in \textbf{bold} and \underline{underlined}.}
\label{tab:sem_kitti_val}
\end{table*}

\begin{figure*}[t]
    \centering
    \includegraphics[width=\linewidth]{imgs/sup_vis1.pdf}
    \caption{Qualitative visualizations on SemanticKITTI validation set. In comparison to previous methods, our method exhibits notably superior and more distinct prediction outcomes, particularly in its ability to distinguish individual instances in complex scenes.
    }
    \label{fig:supp_viz1}
\end{figure*}
\end{document}
